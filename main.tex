\documentclass[12pt]{article}
\usepackage[utf8]{inputenc}
\usepackage[english]{babel}
\usepackage{amsmath}
\usepackage{amssymb}
\usepackage{graphicx}
\usepackage{geometry}
\usepackage{hyperref}
\usepackage{setspace}
\usepackage{float}
\usepackage{natbib}
\geometry{margin=1in}
\setstretch{1.2}

\title{Risk as Causal Pathway Structure: A Topological Framework for Analyzing Systemic Futures}
\author{}
\date{}

\begin{document}
\maketitle

\begin{abstract}
Traditional risk assessment treats risk as a static quantity derived from the product of hazard, exposure, and vulnerability. This approach fails to capture the branching, often non-ergodic futures characteristic of complex Social--Ecological--Technological Systems (SETS). We propose a multiway causal graphs approach based on local transformation rules, in which risk is quantified from the topological shape of possible futures as an alternative to probability-weighted risk aggregation when facing deep uncertainty. Building on graph rewriting systems and network theory, we construct multiway causal graphs that represent all possible system trajectories generated by local transformation rules. We introduce several potential topological metrics to quantify resilience and fragility through the structure of these possibility spaces. Through a comparison of two simple examples (glass and plant), we demonstrate how identical initial conditions produce fundamentally different risk profiles depending on transformation rule topology. 
\end{abstract}

\section{Introduction}

\subsection{Motivation: The Limits of Static Risk Assessment}

The classical risk triplet formulation (\emph{hazard} $\times$ \emph{exposure} $\times$ \emph{vulnerability}) \citep{kaplan1981quantitative, turner2003framework} has proven useful for well-characterized risks with stable probability distributions. However, this framework encounters fundamental limitations when applied to complex Social--Ecological--Technological Systems (SETS) that exhibit emergent behavior, nonlinear dynamics, and deep uncertainty \citep{walker2013deep, lempert2003shaping}. Consider three examples that illustrate these limitations:

First, critical infrastructure networks demonstrate cascading failures where initial disturbances propagate through interdependencies in ways that cannot be predicted from component-level vulnerability assessments alone \citep{buldyrev2010catastrophic}. Several examples highlight systemic impacts emerging from the structure of dependencies rather than from the magnitude of initial perturbations.

Second, ecosystem regime shifts occur when gradual environmental changes suddenly trigger transitions to alternative stable states \citep{scheffer2009critical}. Often loss states cannot be adequately characterized by static vulnerability metrics because the system response is fundamentally path-dependent, exhibiting thresholds and hysteresis effects (i.e. history dependant) that make reversal difficult or impossible once crossed.

Third, sociotechnical transitions—such as energy system or digital transformation—involve lock-in effects \citep{arthur1989competing, david1985clio} where early decisions constrain future options through irreversible commitments. The configuration of infrastructure, institutions, and social practices creates path dependencies that cannot be captured by aggregate risk scores.

These examples share a common feature: risk emerges not from static properties of system components but from the \emph{structure of possible futures} generated by system evolution. Hence, we propose a fundamental shift from static assessment to a generative viewpoint, wherein risk is understood as a property of the multiway causal graph—the ensemble of all possible trajectories that the system might follow.

\subsection{A Topological Approach to Systemic Risk}

This paper develops a framework in which system states are represented as graph configurations, and evolution is governed by local transformation rules that rewrite these configurations. The repeated application of transformation rules generates a multiway causal graph whose structure encodes the space of possible futures. Risk then corresponds to topological properties of this graph: systems with many interconnected pathways exhibit resilience, while systems with narrowing connectivity and few pathways exhibit fragility.

This approach draws on several conceptual foundations. From rewriting systems \citep{wolfram2002}, we adopt the principle that complex behavior can emerge from simple local rules. From network science \citep{buldyrev2010catastrophic}, we take methods for quantifying connectivity and robustness. From resilience theory \citep{holling1973resilience, scheffer2009critical}, we incorporate concepts of adaptive cycles and alternative stable states. From the storylines approach to climate scenarios \citep{shepherd2018storylines, sillmann2021event}, we adopt the emphasis on exploring possible futures rather than assigning precise probabilities. These conceptual threads are unified into a coherent formalism that provides both theoretical insight and practical metrics.

The framework operates at multiple scales depending on analytical goals and data availability. At the conceptual level, it reformulates risk as an emergent property of causal topology rather than as a static aggregate. At the formal level, it provides precise mathematical language for describing system transitions and pathway interactions. At the computational level, when transformation rules are empirically calibrated to specific systems, the framework becomes a simulation tool capable of generating concrete scenario ensembles for decision support.

\subsection{Contributions and Organization}

This work makes three primary contributions. First, we provide a formal framework linking system transformation rules to the topology of possible futures, showing how resilience and fragility emerge from pathway structure rather than from component properties. Second, we introduce four computable topological metrics that quantify decision-relevant features of the possibility space: pathway multiplicity, reconvergence coefficient, topological sensitivity, and spectral gap. Third, we demonstrate through a synthetic comparison how identical initial conditions and disturbance exposures can produce fundamentally different risk profiles depending on transformation rule topology.

The paper proceeds as follows. Section 2 develops the formal framework, defining system configurations, transformation rules, the multiway causal graph, and topological metrics. Section 3 presents a detailed comparison of two synthetic regimes—one dominated by degradation processes, the other incorporating adaptive repair. Section 4 discusses implications for risk assessment practice and identifies limitations and extensions. 

\section{Framework: States, Rules, and Topological Metrics}

\subsection{System Representation as Graph Configurations}

We represent a system state as a graph configuration \(\Sigma = (V, E)\), where \(V\) denotes the set of system components and \(E\) represents the connections, dependencies, and interactions between these components \citep{markolf2018interdependent}. This representation is motivated by three considerations. First, SETS are fundamentally relational, with system behavior emerging from connection patterns as much as from individual element properties. Second, graph representations naturally accommodate heterogeneous components spanning social, ecological, and technological domains within a unified framework. Third, graphs provide intuitive visualization of system structure and evolution, facilitating communication of complex mathematical objects.

For example, in an infrastructure system, nodes might represent power stations, substations, transmission lines, water treatment facilities, and communication hubs, while edges represent dependencies. In an ecosystem, nodes might represent species populations or habitat patches, with edges representing predator-prey relationships, pollination dependencies, or migration corridors. The graph abstraction allows us to reason about these diverse systems using common mathematical language.

\subsection{Transformation Rules as Graph Rewriting Operations}

System evolution is driven by a set of local transformation rules \(R = \{r_1, \ldots, r_n\}\), each of which acts as a rewriting operator on graph configurations. Formally, each rule maps a configuration to a set of possible successor configurations:
\[
    r: \Sigma \rightarrow \{\Sigma'_1, \ldots, \Sigma'_k\}.
\]
The multiplicity of successors (\(k > 1\)) reflects genuine uncertainty about which outcome will occur when the rule is applied, capturing both aleatory uncertainty (inherent stochasticity in system behavior) and epistemic uncertainty (limited knowledge about system response).

At their most fundamental level, transformation rules are purely structural operations that manipulate graph topology through node and edge creation, deletion, or modification. Crucially, the rules themselves are defined solely by their graph rewriting operations—they specify which nodes or edges are added, removed, or modified under what conditions. This separation between rule definition (graph operations) and rule consequences (observable effects) is conceptually important.

Once a rule is applied to a configuration, we can observe and quantify its effects through a signature vector:
\[
\delta \vec{r} = (\Delta U, \Delta E, \Delta C),
\]
where the components represent changes in utility or performance (\(\Delta U\)), energetic, economic, or operational effort (\(\Delta E\)), and constraints on future evolution (\(\Delta C\)). These quantities are not determinants of the rule but rather emergent properties that can be measured after the rule is applied. The same graph operation may produce different observable signatures depending on the configuration to which it is applied and the context-specific interpretation of utility, effort, and constraint.

To summarize, discrete transformation rules specify how system configurations evolve structurally, and we subsequently measure the functional implications of these structural changes.

\subsection{The Multiway Causal Graph}

Beginning from an initial configuration \(\Sigma_0\), we iteratively apply all applicable transformation rules to generate a tree of possible futures. Each node in this tree represents a reachable system configuration, and each edge represents a possible transformation. Because multiple rules may apply to the same configuration, and because individual rules may produce multiple outcomes, the tree branches into a multiway graph capturing the full space of possible trajectories.

Formally, the multiway causal graph \(G = (S, T)\) consists of:
\begin{itemize}
    \item A set \(S\) of system configurations, where each configuration is a graph \(\Sigma = (V, E)\)
    \item A set \(T\) of transitions, where each transition represents a rule application taking one configuration to another
\end{itemize}

This construction differs from conventional scenario analysis in two important ways. First, the multiway graph is generated systematically from explicit transformation rules rather than being constructed ad hoc based on expert judgment about plausible futures (few examples are given Table \ref{rules}).

\begin{table}[H]
\centering
\begin{tabular}{l l c c c}
\hline
Domain & Rule & $\Delta U$ & $\Delta E$ & $\Delta C$ \\
\hline
Infrastructure & Pipe rupture (edge deletion) & $\downarrow$ & $\downarrow$ & $\uparrow$ \\
Infrastructure & Parallel capacity (edge creation) & $\uparrow$ & $\uparrow$ & $\downarrow$ \\
Ecological & Extinction (node deletion) & $\downarrow$ & $\downarrow$ & $\uparrow$ \\
Ecological & Mutualism (edge creation) & $\uparrow$ & $\uparrow$ & $\downarrow$ \\
Social & Tie dissolution (edge deletion) & $\downarrow$ & $\downarrow$ & $\uparrow$ \\
Social & Bridging (edge creation) & $\uparrow$ & $\uparrow$ & $\downarrow$ \\
\hline
\end{tabular}
\caption{Examples of rule signatures for SETS processes across infrastructure, ecological, and social domains. Deletion rules generally reduce utility and increase constraints, while creation rules increase utility at the cost of effort while reducing constraints.}
\label{rules}
\end{table}

This systematic generation ensures logical consistency—all represented futures are reachable through specified mechanisms—while also enabling exploration of unexpected combinations that human judgment might overlook. Second, the multiway graph explicitly represents the \emph{relationships} between scenarios, showing how different pathways branch from common ancestors or converge toward common descendants. This relational structure enables topological analysis of the possibility space.

\subsection{Topological Metrics for Resilience and Fragility}

The structure of the multiway causal graph encodes information about system resilience and fragility that can be quantified through topological metrics. We introduce four metrics designed to capture decision-relevant features of the possibility space.

\subsubsection{Pathway Multiplicity (\(N_{\mathrm{paths}}\))}

The simplest measure of possibility space richness is the number of distinct directed pathways from the initial configuration to terminal configurations or to a fixed temporal horizon. Formally:
\[
N_{\mathrm{paths}} = |\{\gamma: \Sigma_0 \to \Sigma_T \mid \gamma \text{ is a directed path in } G\}|.
\]
Higher \(N_{\mathrm{paths}}\) indicates greater diversity of possible futures, suggesting that the system retains options and has not been forced into a narrow channel. Conversely, declining \(N_{\mathrm{paths}}\) over time signals narrowing possibility spaces and potential lock-in. This metric provides a first-order indicator of the system's adaptive capacity: systems with many available pathways can explore diverse responses to perturbations, while systems with few pathways have limited options for adaptation.

\subsubsection{Reconvergence Coefficient (\(\rho\))}

While pathway multiplicity measures how many futures are possible, reconvergence measures how often distinct pathways reunite. We define the reconvergence coefficient as the fraction of configurations in the multiway graph that have multiple incoming edges from distinct ancestor configurations:
\[
\rho = \frac{\text{number of configurations with in-degree} \geq 2}{\text{total number of configurations}}.
\]
High \(\rho\) indicates that different disturbance sequences often lead to equivalent outcomes, suggesting structural degeneracy and functional redundancy. This is a form of resilience: even if the system follows different trajectories through configuration space, it frequently returns to functionally equivalent states. Low \(\rho\), conversely, indicates that pathways rarely reconnect, meaning that early disturbances permanently constrain later possibilities through path dependence and lock-in.

The reconvergence coefficient captures a subtle but important aspect of resilience. A system might have many possible pathways (\(N_{\mathrm{paths}}\)) but exhibit low reconvergence if these pathways diverge permanently. Such a system has diversity but not robustness—different disturbance sequences lead to permanently different outcomes. A truly resilient system combines pathway diversity with reconvergence, maintaining options while demonstrating insensitivity to disturbance sequencing.

\subsubsection{Topological Sensitivity (\(S(\theta)\))}

System robustness can also be assessed by examining how the multiway graph structure changes when parameters or forcing conditions vary. We define topological sensitivity as the rate of change of graph properties with respect to a parameter \(\theta\):
\[
S(\theta) = \left|\frac{\partial N_{\mathrm{paths}}}{\partial \theta}\right| + \left|\frac{\partial \rho}{\partial \theta}\right| + \left|\frac{\partial \lambda_2}{\partial \theta}\right|,
\]
where \(\lambda_2\) is the spectral gap (defined below) and the derivatives are approximated by finite differences between multiway graphs generated at nearby parameter values. High \(S(\theta)\) indicates that small changes in conditions produce large changes in the structure of the possibility space, suggesting proximity to critical transitions. Low \(S(\theta)\) indicates structural stability, where the topology of possible futures remains consistent across parameter variations.

This metric provides an early warning system for regime shifts. Unlike conventional early warning signals that detect changes in system state \citep{scheffer2009critical}, topological sensitivity detects changes in the \emph{structure of possibilities}. A system might currently maintain high performance while exhibiting rising topological sensitivity, signaling that the space of possible futures is becoming increasingly fragile even though no functional degradation has yet occurred.

\subsubsection{Spectral Gap (\(\lambda_2\))}

Finally, we quantify the overall connectivity of the possibility space through the spectral gap of the multiway graph's Laplacian matrix. The normalized graph Laplacian is defined as:

\[
\mathcal{L} = I - D^{-1/2}AD^{-1/2}\
\]

where \(A\) is the adjacency matrix and \(D\) is the degree matrix. 
The eigenvalues \(0 = \lambda_1 \leq \lambda_2 \leq \cdots \leq \lambda_n \leq 2\) of \(\mathcal{L}\) encode connectivity information, with the second eigenvalue \(\lambda_2\) (the spectral gap) providing a particularly useful measure of algebraic connectivity \citep{fiedler1973algebraic}.

For connected graphs, \(\lambda_2 > 0\), with higher values indicating stronger connectivity. The spectral gap captures how easily disturbances can propagate through the network of possibilities and how robust the graph is to node or edge removal. Systems approaching \(\lambda_2 \to 0\) are fragmenting into disconnected regions of possibility space, indicating loss of adaptability. Systems maintaining high \(\lambda_2\) preserve connectivity between different regions of possibility space, enabling flexible navigation between alternative futures.

Together, these four metrics—\(N_{\mathrm{paths}}\), \(\rho\), \(S(\theta)\), and \(\lambda_2\)—provide complementary perspectives on the topology of possible futures. A resilient system exhibits high pathway multiplicity (many options), high reconvergence (frequent pathway reconnection), low topological sensitivity (structural stability), and high spectral gap (strong connectivity).

\section{Case Study: Glass versus Plant Regimes}

\subsection{Motivation and Design}

To demonstrate how transformation rule topology shapes risk profiles, we construct a controlled comparison of two synthetic regimes representing opposite ends of a resilience spectrum. The ``Glass'' regime is dominated by degradation processes without compensating repair mechanisms, while the ``Plant'' regime incorporates adaptive repair processes alongside degradation. This comparison is deliberately simplified to isolate the effects of transformation rule structure on multiway graph topology. Real systems typically occupy intermediate positions on this spectrum, exhibiting some combination of degradation and repair processes.

Both regimes begin from an identical initial configuration: a simple triangle graph representing a minimal connected system. This deliberately minimal starting point allows us to clearly trace how different transformation rules shape the evolution of complexity and connectivity. Both regimes are also subject to the same universal degradation process, modeled as edge deletion representing loss of connections due to aging, disturbance, or resource limitation. The key distinction lies in the presence or absence of repair mechanisms that can counteract degradation.

\subsection{Transformation Rules}

\subsubsection{Rule 1: Entropy (Universal Degradation)}

Both regimes share a common degradation rule that removes edges from the configuration:
\[
r_{\mathrm{entropy}}: (V, E) \to (V, E \setminus \{e\}) \quad \text{for each } e \in E.
\]
This rule can be applied to any existing edge, and when applied, it removes that edge from the graph. If multiple edges exist, the rule generates multiple successor configurations (one for each possible edge to delete), reflecting uncertainty about which connection will fail first. The entropy rule represents ubiquitous processes of degradation, deterioration, and disturbance that affect all systems.

Once an edge is deleted, nodes may become isolated if that edge provided their only connection to the network. The rule continues to be applicable as long as any edges remain; once all edges are deleted, the system reaches an absorbing state consisting of isolated nodes with no connections. This terminal configuration represents complete system collapse.

\subsubsection{Rule 2: Repair (Plant Regime Only)}

The Plant regime incorporates an additional repair rule that can restore or create new connections:
\[
r_{\mathrm{repair}}: (V, E) \to (V, E \cup \{(u,v)\}) \quad \text{if } u,v \text{ share a common neighbor}.
\]
This rule identifies ``open triangles''—configurations where two nodes are both connected to a third node but not directly to each other—and closes them by adding the missing edge. This mechanism represents adaptive processes that recognize structural vulnerabilities (the open triangle as a weak configuration) and respond by creating redundant pathways (the completed triangle as a robust configuration).

The repair rule embodies several important principles. First, it operates locally based on existing structure rather than requiring global coordination or external intervention. Second, it creates structural degeneracy by adding alternative pathways between nodes. Third, it can counteract but not fully reverse degradation—deleted edges may be replaced by different edges, leading to configurations that are functionally equivalent but structurally distinct from the initial state.

The interplay between entropy and repair rules generates rich dynamics. Degradation creates open triangles by breaking edges in previously complete triangles. Repair closes open triangles by adding edges. The system's trajectory through configuration space depends on the relative rates and sequencing of these competing processes.

\subsection{Comparative Results}

We generated multiway causal graphs for both regimes starting from the same initial triangle configuration and allowing evolution for six temporal steps (Figure \ref{fig:graphs}). This limited horizon was chosen to keep the computational analysis tractable while still capturing the key topological differences between regimes. Table~\ref{tab:metrics} presents quantitative comparisons of the four topological metrics.

\begin{figure}[H]
    \centering
    \includegraphics[width=1\linewidth]{multiway_futures_comparison.png}
    \caption{Comparative evolution of the Glass (left) and Plant (right) multiway causal graphs over equivalent time horizons. The Glass system exhibits a collapsing topology where all paths converge to a single failure state, showing high early diversity that progressively narrows to a unique terminal configuration. The Plant system exhibits a rich, recombining topology with high degeneracy, maintaining multiple viable configurations through repair cycles that prevent convergence to failure. Despite identical initial hazard and exposure conditions, the difference in transformation rules produces radically different possibility spaces, demonstrating that risk emerges from causal topology rather than from static vulnerability metrics.}
    \label{fig:graphs}
\end{figure}

\begin{table}[H]
\centering
\begin{tabular}{|l|c|c|l|}
\hline
\textbf{Metric} & \textbf{Glass} & \textbf{Plant} & \textbf{Interpretation} \\
\hline
Total States (Nodes) & 8 & 12 & Configuration space size \\
$N_{\mathrm{paths}}$ & 6 & 10 & Pathway multiplicity \\
$\rho$ (Reconvergence) & 0.500 & 0.667 & Structural degeneracy \\
$S(\theta)$ (Sensitivity) & 0.028 & 0.015 & Topological transition risk \\
$\lambda_2$ (Spectral Gap) & 0.6667 & 0.3333 & Algebraic connectivity \\
\hline
\end{tabular}
\caption{Quantitative topological metrics comparing the Glass and Plant regimes computed from the multiway causal graphs at simulation depth of 6 steps. The Glass system shows fewer pathways (6 vs 10), moderate reconvergence (0.500 vs 0.667), higher topological sensitivity, and higher spectral gap reflecting simpler topology. The Plant system maintains higher pathway diversity, more frequent reconvergence through repair mechanisms, lower sensitivity indicating structural stability, and moderate connectivity. These metrics quantify the distinction between brittle and resilient systems through precise topological features of the multiway causal graph.}
\label{tab:metrics}
\end{table}

These quantitative differences reflect fundamental distinctions in the topology of possible futures. The Glass regime exhibits lower pathway multiplicity (\(N_{\mathrm{paths}} = 6\) versus 10), indicating that degradation without repair progressively narrows the space of possibilities. The system retains some pathway diversity because different sequences of edge deletions lead to different intermediate configurations, but this diversity decreases monotonically as edges are lost and options close off.

The reconvergence coefficient reveals a more subtle difference. Both regimes show substantial reconvergence (\(\rho = 0.500\) for Glass, 0.667 for Plant), but the meaning differs. In the Glass regime, reconvergence occurs because different degradation sequences eventually lead to the same depleted configurations—pathways converge toward shared failure states. This represents the worst possible form of reconvergence: insensitivity to disturbance details arising from collapse of all diversity rather than from adaptive compensation.

In contrast, the Plant regime's higher reconvergence (\(\rho = 0.667\)) reflects functional equivalence achieved through active repair. Different disturbance sequences trigger different repair responses, but these alternative repair pathways frequently reunite at configurations that are structurally distinct yet functionally equivalent. This nontrivial reconvergence provides genuine resilience because the convergent states maintain viability rather than representing shared failure.

The topological sensitivity results (\(S(\theta) = 0.028\) for Glass versus 0.015 for Plant) indicate that the Glass regime is more sensitive to parameter variations. This might initially seem counterintuitive—shouldn't the more complex adaptive regime be more sensitive? However, the result makes sense when we recognize that the Glass regime is following a narrow channel toward inevitable collapse. Small parameter changes can shift which edges fail first, producing relatively large changes in intermediate pathway structure even though all paths lead to the same terminal state. The Plant regime, with its richer set of compensating mechanisms, maintains more consistent possibility space structure across parameter variations.

Finally, the spectral gap comparison (\(\lambda_2 = 0.6667\) for Glass versus 0.3333 for Plant) reflects the relative simplicity of the Glass regime's topology. The unidirectional flow toward collapse creates a more linearly organized possibility space with fewer cross-connections, resulting in higher spectral gap. The Plant regime's more complex topology, with multiple repair-generated pathways creating mesh-like connectivity, results in lower spectral gap. This counterintuitive result highlights that spectral gap must be interpreted in context—high \(\lambda_2\) is not always beneficial if it arises from oversimplified topology rather than from robust connectivity.

\subsection{Causal Invariance, Closure, and Lock-In}

Beyond the aggregate metrics, the two regimes exhibit fundamentally different properties regarding causal invariance—the extent to which different transformation sequences lead to equivalent outcomes. The Glass regime exhibits trivial causal invariance through collapse: all pathways eventually reach the same absorbing failure state regardless of the sequence of edge deletions. However, this represents the worst possible form of invariance, where insensitivity to pathway details arises from the elimination of all functional diversity rather than from adaptive compensation.

The Plant regime exhibits nontrivial causal invariance through functional equivalence classes. Multiple distinct transformation sequences reunite into configurations that are structurally different but functionally equivalent, providing the same performance and adaptive capacity. For example, after suffering edge deletions in different locations, the system may form compensating connections through different repair mechanisms, arriving at distinct graph configurations that nonetheless support equivalent function. This nontrivial closure provides genuine resilience because the reconvergent states maintain viability rather than representing shared failure.

The Glass regime also demonstrates lock-in effects characteristic of systems lacking adequate repair mechanisms. Edge deletion is inherently non-commutative with respect to potential repair rules (which are absent). Once an edge is deleted, any functionality depending on that connection is permanently lost, closing off entire branches of the possibility space. The system evolves as a pure divergence tree where pathways never reconnect except at failure states, and the configuration space progressively fragments into isolated, unreachable regions. This loss of connectivity manifests as lock-in effects \citep{arthur1989competing, david1985clio} familiar from economic systems with increasing returns, technological systems with compatibility requirements, and ecological systems approaching extinction thresholds.

The Plant regime avoids lock-in through repair rules that provide partial reversibility and alternative pathways to functionally equivalent states. While specific transformation sequences remain path-dependent in their details, the functional consequences exhibit robustness to ordering variations. This flexibility emerges from structural degeneracy, where multiple configurations can satisfy the same functional requirements, combined with repair mechanisms that enable navigation between these functionally equivalent but structurally distinct states.

\section{Discussion: Implications and Limitations}

\subsection{Implications for Risk Assessment Practice}

The Glass-Plant comparison demonstrates how identical initial conditions and disturbance exposures can produce radically different risk profiles depending on the topology of transformation rules. This finding underscores the limitations of conventional risk assessment approaches that focus on hazard and exposure while treating vulnerability as a static property. Our analysis reveals that system evolution is fundamentally shaped by the balance between degradation and repair processes, and that resilience emerges from the structural properties of the rule set rather than from any single component or parameter.

The topological metrics introduced here translate abstract resilience concepts into quantifiable features of the possibility space that can guide policy design, infrastructure planning, and ecosystem management. Systems exhibiting declining \(N_{\mathrm{paths}}\) and \(\lambda_2\) signal deteriorating adaptive capacity even before functional performance degrades, enabling proactive intervention before crises occur. Conversely, systems maintaining high pathway diversity and connectivity demonstrate robustness to perturbations. Spikes in topological sensitivity \(S(\theta)\) provide early warning of critical transitions, indicating when the system approaches thresholds where small changes in forcing or parameters could trigger regime shifts in the structure of the possibility space.

The framework also provides practical guidance for intervention design. Policies that preserve or create structural degeneracy—multiple pathways to desired outcomes—enhance resilience by maintaining options. Interventions that reduce \(\Delta C\) expand future options, while those that increase \(\Delta C\) may offer short-term benefits at the cost of long-term flexibility. By explicitly representing the space of possible futures rather than aggregating over scenarios, the multiway approach enables robust decision-making that accounts for deep uncertainty without requiring precise probability assignments \citep{lempert2003shaping, walker2013deep}.

Consider infrastructure planning as a concrete application. Conventional approaches might assess risk by estimating failure probabilities for individual components and computing expected losses. The topological approach would instead examine how infrastructure design shapes the multiway graph of possible futures. A design that maintains high \(N_{\mathrm{paths}}\) and \(\rho\) preserves multiple pathways to service restoration after disturbances, exhibiting resilience through redundancy and adaptability. A design with low \(N_{\mathrm{paths}}\) and \(\rho\) may perform well under normal conditions but faces catastrophic failure under disturbances because few adaptation pathways exist.

\subsection{Relation to Existing Frameworks}

This work connects to several existing research traditions while offering a distinct perspective. Compared to resilience theory \citep{holling1973resilience, scheffer2009critical}, we provide formal mathematical language for concepts such as adaptive cycles and regime shifts, representing them as topological features of multiway causal graphs. The reconvergence coefficient (\(\rho\)) formalizes the idea of ``panarchy''—the cross-scale interactions that can restore function after disturbances—while topological sensitivity (\(S(\theta)\)) provides quantitative early warning of approaching thresholds.

Compared to network robustness analysis \citep{buldyrev2010catastrophic}, we shift focus from static network properties to the dynamics of possibility spaces. While traditional approaches examine how networks fail under node or edge removal, we examine how the space of possible futures evolves under transformation rules. This dynamic perspective reveals that resilience depends not only on current connectivity but on the capacity to generate new connections through repair mechanisms.

Compared to scenario analysis and storylines approaches \citep{shepherd2018storylines, sillmann2021event}, we provide systematic generation of scenario ensembles from explicit transformation rules rather than relying on expert judgment to select representative futures. The multiway graph makes explicit the relationships between scenarios, enabling analysis of how different storylines branch from common ancestors or converge toward common futures. This relational structure supports more rigorous reasoning about scenario diversity and representativeness.

Compared to dynamic systems modeling and simulation, we emphasize topological analysis of entire possibility spaces rather than sampling individual trajectories. While conventional simulations generate large ensembles of individual runs and aggregate statistics, the multiway graph represents the logical structure of all possible trajectories simultaneously. This shift from sampling to exhaustive enumeration (within computational limits) provides complementary insights, particularly about structural features such as pathway connectivity and reconvergence that might be difficult to detect from trajectory samples alone.

\subsection{Limitations and Extensions}

Several limitations of the current framework deserve acknowledgment. First, the approach is currently demonstrated through synthetic examples rather than calibrated to empirical systems. Translating the framework into practical decision support tools will require methods for eliciting transformation rules from empirical data, stakeholder knowledge, and existing models. This calibration challenge is substantial but not unique to our approach—all scenario-based methods face questions about how to ensure that explored futures are relevant and plausible.

Second, the computational cost of exhaustive multiway graph generation grows rapidly with the number of transformation rules and the time horizon considered. For complex real-world systems, complete enumeration of all possible futures quickly becomes intractable. However, several strategies can manage this complexity. Coarse-graining can group configurations into equivalence classes based on functional similarity rather than structural identity. Sampling methods can explore representative subsets of the possibility space rather than generating it exhaustively. These computational strategies would require further development.

Third, the framework currently treats transformation rules as given and focuses on analyzing their consequences. An important extension would develop methods for designing transformation rules to achieve desired topological properties. For example, what repair mechanisms would maximize reconvergence while minimizing resource costs? What combinations of transformation rules produce possibility spaces that are both diverse and navigable? These inverse design problems represent an important direction for future research.

Fourth, the current framework does not explicitly incorporate stakeholder preferences, values, or contested knowledge. Different stakeholders may disagree about which configurations are desirable, which transformation rules are plausible, or which timeframes are relevant. Extensions might represent these disagreements explicitly within the multiway graph framework, perhaps by generating multiple graphs corresponding to different stakeholder perspectives and analyzing their similarities and differences.

Fifth, while we have focused on topological metrics computed from the multiway graph structure, the framework could be extended to incorporate other sources of information. For example, if probability distributions over transformation rule applications were available, these could be used to weight pathways and compute probability-weighted metrics. Similarly, if detailed performance models were available for specific configurations, these could be used to evaluate the functional consequences of different topological structures. Such extensions would create bridges between the purely topological approach developed here and more conventional quantitative risk assessment methods.

Despite these limitations, the framework provides a conceptual and mathematical foundation for reasoning about systemic risk that addresses important gaps in existing approaches. The emphasis on possibility space topology, the explicit representation of pathway structure, and the introduction of decision-relevant topological metrics represent distinct contributions that complement rather than replace existing methods.

\section{Conclusion}

This work proposes a fundamental reconceptualization of systemic risk as a property of the topology of possible futures rather than as a static scalar derived from component vulnerabilities. The multiway causal graph framework provides a mathematical structure in which resilience corresponds to structural degeneracy and adaptive reconvergence, while fragility corresponds to narrowing connectivity and irreversible lock-in. This shift from static to dynamic, from scalar to topological, and from aggregate to causal enables new forms of reasoning about systemic risk under deep uncertainty.

The topological metrics introduced—pathway multiplicity, reconvergence coefficient, topological sensitivity, and spectral gap—provide decision-relevant indicators aligned with emerging approaches such as storylines \citep{shepherd2018storylines, sillmann2021event} and robust decision-making under deep uncertainty \citep{walker2013deep, lempert2003shaping}. These metrics translate abstract resilience concepts into quantifiable features of the possibility space that can guide robust decisions when precise probabilities are unavailable or unreliable.

The Glass-Plant comparison demonstrates that transformation rule topology fundamentally shapes risk profiles even when initial conditions, system components, and disturbance exposures are held constant. Systems without adequate repair mechanisms (Glass) exhibit inexorable narrowing of possibilities and lock-in toward failure states, while systems with compensating mechanisms (Plant) maintain diverse, interconnected possibility spaces enabling adaptive response. This finding suggests that resilience interventions should focus not only on reducing exposure to disturbances or hardening individual components but also on fostering structural degeneracy and repair mechanisms that preserve options and enable adaptation.

The multiway causal graph perspective offers not merely a technical tool but a conceptual lens through which to understand how systems navigate uncertain futures. By representing risk as emerging from the structure of possibilities rather than from static properties, this framework aligns with growing recognition that deep uncertainty demands approaches emphasizing robust adaptation over precise prediction \citep{lempert2003shaping, walker2013deep}. As humanity confronts increasingly complex, interconnected challenges from climate change \citep{lenton2008tipping} to technological disruption to social transformation, frameworks that capture the branching, path-dependent, and fundamentally uncertain nature of systemic evolution become essential for navigating toward desirable futures.

Important next steps include empirical calibration of transformation rules for specific systems and computational methods for managing complexity in large-scale applications. Despite current limitations, the framework provides a principled foundation for a paradigm shift in risk assessment.

\bibliographystyle{plainnat}
\bibliography{references}

\end{document}