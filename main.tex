\documentclass[11pt]{article}
\usepackage{graphicx}
\usepackage{amsmath}
\usepackage{amssymb}
\usepackage{amsthm}
\usepackage{hyperref}
\usepackage{geometry}
\geometry{margin=1in}

\theoremstyle{definition}
\newtheorem{definition}{Definition}
\newtheorem{proposition}{Proposition}

\title{Risk as Causal Pathway Structure:\\A New Epistemic Foundation}

\author{Author Names\\Institution}
\date{}

\begin{document}
\maketitle

\begin{abstract}
We propose a foundational reconceptualization of risk science. Rather than treating risk as the product of static factors (hazard, exposure, vulnerability), we view risk as emerging from the structure of branching causal pathways through which systems evolve. This path-distributional perspective reveals that fragility arises not merely from having few viable futures, but from the topology of how those futures relate to each other. We distinguish between redundancy, where multiple pathways share common failure modes, and degeneracy, where pathways fail independently. By representing possible futures as multiway causal graphs where nodes are complete system configurations, we make path dependence, structural coupling, and topological transitions visible and quantifiable. This framework provides a new epistemic foundation for understanding adaptive capacity, detecting approaching tipping points, and distinguishing apparent from genuine robustness.
\end{abstract}

\section{Introduction: The Limits of Classical Risk}

Traditional risk analysis has long relied on a multiplicative framework:
\begin{equation}
\text{Risk} = \text{Hazard} \times \text{Exposure} \times \text{Vulnerability}.
\end{equation}

This formula multiplies the severity of a potential threat, the degree to which something is in harm's way, and the susceptibility to damage. While this triplet formulation has proven useful for static risk assessment \cite{kaplan1981quantitative}, it faces fundamental limitations when analyzing systems that evolve under deep uncertainty.

The key limitations reflect a deeper epistemological problem. Classical risk thinking treats systems as static objects with fixed properties that can be measured and multiplied. Real systems are dynamic processes that evolve through time along branching pathways shaped by decisions, constraints, and optimization pressures (i.e. non-ergodic systems). We propose a fundamental reconceptualization. Risk emerges from the structure of the distribution over possible system trajectories, as shaped by optimization under constraints. This shifts the question from "what is the risk of event X?" to "what is the topology of accessible futures, and how constrained or coupled are the pathways that lead there?" This perspective treats risk as a geometric property of pathway space rather than a scalar product of independent factors.

\section{Trajectories, Optimization, and the Action Principle}

Inspired by path integral formulation in quantum mechanics, the  and recent work on climate storylines, we treat each possible evolution of a system as a trajectory through configuration space. A trajectory represents a complete narrative from present state to future state, a storyline describing one way the system could unfold over time.

\subsection{The Action Functional}

For each trajectory, we define an action functional that captures the fundamental trade-offs systems face:

\begin{definition}[Action Functional]
For a trajectory $\gamma$ from initial state at time $t_0$ to final state at time $t_1$, the action functional is:
\begin{equation}
J[\gamma] = \text{Effort}[\gamma] - \text{Utility}[\gamma] + \text{Constraints}[\gamma].
\end{equation}
\end{definition}

This functional integrates three essential aspects. Effort represents resources, energy, or costs expended along the trajectory. This might be adaptation investment in climate systems, construction costs in infrastructure, or metabolic expenditure in ecosystems. Utility captures benefits, rewards, or value gained. This could be avoided climate damages, throughput efficiency in networks, or ecosystem productivity. Constraints impose penalties for violating system limits or encountering hazards, such as crossing tipping points, exceeding capacities, or triggering extinctions.

The action functional differs from the classical action in physics. Rather than integrating kinetic and potential energy, it integrates decision-theoretic trade-offs relevant to risk management. This generalization makes the framework applicable beyond physical systems to encompass policy decisions, ecological dynamics, and economic evolution.

\subsection{The Optimization Principle}

The universality of optimization principles across complex systems provides strong justification for an action-based framework. Ecosystems minimize energy expenditure while maximizing resource gain through optimal foraging. Financial systems balance return against risk following variational principles. Metabolic networks minimize entropy production while maintaining function. Neural systems minimize prediction error through the free energy principle \cite{friston2010free}. Physical systems follow the principle of least action.

This cross-domain convergence suggests that systems operating under uncertainty universally evolve through optimization of action-like functionals. Systems seek paths that balance effort against utility while respecting constraints. This is not an assumption we impose but an empirical regularity we observe across scales and domains.

When systems face multiple possible trajectories with different action scores, optimization processes increase the probability of low-action trajectories relative to high-action trajectories. The relationship takes an exponential form familiar from statistical mechanics:

\begin{equation}
P(\gamma) \propto e^{-\beta J[\gamma]} \cdot P_0(\gamma)
\end{equation}

where $P_0(\gamma)$ represents baseline tendencies before optimization, and $\beta$ controls optimization strength. Higher $\beta$ indicates stronger concentration of probability on the lowest-action path. Lower $\beta$ indicates broader exploration across many paths.

This formulation reveals a crucial insight. Risk analysis must account for how optimization concentrates probability on particular futures. Systems do not randomly sample from all possible trajectories. They actively navigate toward paths that minimize action, reshaping the distribution of accessible futures in the process.

\section{Multiway Causal Graphs: Visualizing Configuration Evolution}

To make trajectory distributions concrete and computable, we introduce a representation inspired by the Wolfram Physics Project's multiway systems, specifically adapted for the context of complex adaptive systems. The key insight is recognizing that system states are not simple scalar values but complete configurations with internal structure.

\subsection{Configurations as Social-Ecological-Technological Systems (SETS)}

Traditional dynamical systems represent states as vectors of numbers. For complex anthropogenic systems, this is insufficient. A state must capture the coupled nature of human, natural, and physical components. We therefore adopt the Social-Ecological-Technological Systems (SETS) framework \cite{mcphearson2016scientists, markolf2018interdependent}.

\begin{definition}[SETS Configuration]
A system configuration $\Sigma$ is a complete specification of the tri-variate coupling between Social (S), Ecological (E), and Technological (T) domains at one moment in time.
\end{definition}

In this view, a "state" is never just a level of infrastructure service (T). It is that infrastructure state coupled with the ecological conditions (E) it relies upon and the social/institutional structures (S) that manage it \cite{krueger2019requisite}. This richness matters profoundly for risk assessment because two systems might share the same aggregate statistics (e.g., identical GDP or water output) yet differ fundamentally in the structural coupling of their SETS components. One might be brittle due to rigid institutional lock-in \cite{markolf2018interdependent}, while the other possesses requisite resilience through loose coupling.

\subsection{The Coupling Mechanism: Action in SETS Coordinates}

The action functional $J[\gamma]$ serves as the integration layer that couples the Social, Ecological, and Technological domains. Rather than treating these as separate optimization problems, the SETS framework recognizes that the terms of the functional themselves force the domains to interact. The functional $J = \text{Effort} - \text{Utility} + \text{Constraints}$ operates on the aggregate system:

\begin{itemize}
    \item \textbf{Effort} aggregates the metabolic and thermodynamic costs of maintaining system complexity across domains. This captures the trade-offs between technological maintenance (T), social investment and political capital (S), and biological energy expenditure (E). A reduction in effort in one domain (e.g., deferred maintenance) often transfers load to another (e.g., increased social stress), preserving the total action principle.
    \item \textbf{Utility} aggregates the system's functional performance. This includes ecosystem services (E), infrastructure throughput and reliability (T), and human wellbeing and equity (S). The functional forces the system to navigate pathways that balance these often competing forms of value.
    \item \textbf{Constraints} represent the hard boundaries of the phase space. These are the non-negotiable limits: planetary boundaries and carrying capacities (E), physical material limits (T), and minimum viable social cohesion (S).
\end{itemize}

Risk emerges because optimizing for one term often drives the system toward the boundaries of another. For example, a trajectory that minimizes \textit{Effort} (by using cheap, fossil-fuel-intensive technology) may initially appear favorable (low action) but inevitably intersects with severe Ecological \textit{Constraints} (climate tipping points). The action functional captures this dynamic coupling, revealing how local optimization in one domain generates systemic risk through the shared constraint landscape.

\subsection{The Multiway Causal Graph}

In defining the evolution of these configurations, we must distinguish between two related topological structures, following the formalism of the Wolfram Physics Project \cite{wolfram2020class}.

\begin{definition}[Multiway Evolution vs. Causal Graph]
The \textbf{Multiway Evolution Graph} consists of nodes representing complete system configurations (states) and edges representing possible updates (transitions). However, the deeper structure is the \textbf{Multiway Causal Graph}, where nodes are \textit{updating events} and edges represent \textit{causal dependencies} between those events.
\end{definition}

In the risk context, the Multiway Evolution Graph maps the "space of possible futures"—showing which system states can transition to which others. The Multiway Causal Graph maps the "structure of necessity"—showing which prior decisions or failures are necessary prerequisites for a future event to occur.

For example, in a water distribution network:
\begin{itemize}
    \item The \textit{Evolution Graph} shows State A (old pipes) transitioning to State B (renewed pipes).
    \item The \textit{Causal Graph} shows that the event "Renew Pipes" causally depends on the event "Approve Budget," which causally depends on "Detect Leak."
\end{itemize}

The multiway causal graph makes visible what traditional risk analysis misses: the causal density and light-cone structure of events. A "risky" future is not just one with low utility, but one reachable through a causal graph with high fan-in (inevitable) or low causal depth (imminent).

\subsection{The Generative Grammar of Risk: Geometry over State}

A crucial distinction in this framework is that \textbf{risk is not a property of a state (node), but a property of the graph's global topology}. 

In classical risk, we assign a risk value to a specific outcome (e.g., "Probability of Flood = 1\%"). In the multiway causal framework, risk is the \textit{shape} of the possibility space itself. It is a holistic geometric feature, not a local variable. 

\begin{itemize}
    \item \textbf{Fragility as a Funnel:} If the causal graph exhibits a "funnel" shape—where millions of diverse pasts converge onto a single unavoidable future configuration—that system is fragile, regardless of how "safe" that future state appears. The lack of optionality is the risk.
    \item \textbf{Robustness as Geodesic Bundling:} If the graph shows a "bundle" of parallel, non-intersecting causal paths that reach similar functional outcomes via different causal histories, the system possesses genuine degeneracy and robustness.
\end{itemize}

This geometry emerges bottom-up from the iterative application of simple local rules (a "generative grammar" \cite{wolfram2002}), rather than being imposed top-down. We do not model the "risk of failure"; we model the rules of interaction. The risk is the resulting abstract shape—the "shadow" cast by the system's causal evolution.

\subsection{Trajectories as Graph Paths}

There is a precise correspondence between trajectories in the abstract action formulation and paths through the multiway graph. Each path through the graph, starting from the present configuration and extending forward in time through branching decisions and events, represents one complete trajectory with its associated action score computed by summing edge costs along the path.

This representation provides several advantages for risk assessment. It makes path dependence explicit through graph connectivity. Once a system passes through a particular configuration, only edges emanating from that node remain accessible. Prior decisions have closed off other branches. It enables efficient computation through dynamic programming algorithms that exploit graph structure rather than enumerating all possible trajectories. It offers visual interpretability where stakeholders can literally see branching points where decisions matter most and bottlenecks where vulnerability concentrates.

\section{Measuring Fragility: Beyond Counting Paths}

The simplest fragility metric counts how many viable paths lead from present to acceptable futures. A system with many paths appears robust, while a system with few paths appears fragile. This intuition captures something important but misses crucial distinctions that become visible through deeper analysis of pathway topology.

\subsection{Effective Number of Paths}

Rather than simply counting paths, we quantify diversity through entropy \cite{shannon1948mathematical}:

\begin{equation}
H = -\sum_{\gamma} P(\gamma) \ln P(\gamma)
\end{equation}

where the sum runs over all paths through the multiway graph. The exponential of entropy gives an effective path count:

\begin{equation}
N_{\text{eff}} = e^H
\end{equation}

This number represents how many equally probable paths would generate the observed entropy. A system with $N_{\text{eff}} = 1$ is locked onto a single trajectory. A system with $N_{\text{eff}} = 100$ behaves as if it has one hundred equiprobable futures. This metric is model-free and interpretable regardless of the underlying concentration mechanism.

However, effective path count alone provides an incomplete picture. Two systems might both show $N_{\text{eff}} = 100$, suggesting equivalent robustness, yet differ fundamentally in the nature of their pathway diversity.

\subsection{Redundancy versus Degeneracy}

This distinction reveals why some systems are genuinely robust while others merely appear so. Redundancy means having multiple copies of essentially the same solution. Multiple pathways through the multiway graph exist, but they share similar structural properties and fail together under stress. Degeneracy means having multiple structurally different pathways that achieve the same functional outcome through independent mechanisms.

Consider an infrastructure network where three parallel pipes connect the same two locations. The multiway graph shows three pathways, contributing to a higher path count. However, all three pipes traverse the same geographic corridor and share exposure to the same flood risk. A single event can eliminate all three pathways simultaneously because they are structurally coupled. This is redundancy, which provides apparent but not genuine robustness.

Contrast this with a network where three completely different configurations achieve the same throughput capacity through distinct routes using separate physical infrastructure. One routes flow through northern pipes, another through southern pipes, a third through a distributed mesh. These pathways diverge in configuration space. They fail independently because they operate through different mechanisms and respond differently to perturbations. This is degeneracy, which provides genuine robustness.

The multiway graph representation makes this distinction visible. Redundant pathways remain close in configuration space, passing through nodes with similar structural properties. Degenerate pathways diverge widely, passing through configurations with different topologies, spatial patterns, or compositions. Visual inspection of the graph reveals whether apparent pathway diversity reflects true independence or illusory correlation.

\subsection{Topological Sensitivity and Early Warnings}

A system might currently possess many viable pathways yet be structurally unstable, poised to undergo rapid reorganization that eliminates most options \cite{bak1987self}. This latent fragility requires measuring not just pathway count but sensitivity of graph topology to perturbations.

Fisher information quantifies this sensitivity. In our framework, it measures how rapidly the probability distribution over pathways changes when system parameters shift slightly. High Fisher information indicates that small changes in constraints, optimization pressure, or environmental conditions cause large reorganization of which pathways remain viable. The multiway graph undergoes topological phase transitions where connectivity patterns, bottleneck locations, and pathway diversity transform dramatically.

This provides an early warning signal complementary to pathway counting. A system might have high $N_{\text{eff}}$ today but also high Fisher information, signaling that this diversity is structurally fragile. The multiway graph currently appears richly connected with many branches, but slight perturbations cause large portions to effectively disconnect, converting many-path regions into narrow bottlenecks or fragmenting the configuration space into isolated clusters.

Consider an ecosystem represented as a multiway graph where nodes are complete spatial vegetation patterns. Currently, many successional pathways exist from disturbed state to productive mature forest, suggesting high adaptive capacity. However, slight increases in fire frequency or invasive species pressure cause Fisher information to spike. Small parameter changes trigger topological transitions where most pathways funnel through a single degraded configuration or the graph fragments into disconnected components representing alternative stable states with no transitions between them. The system appears robust by path count alone but is actually near a critical point.

The connection to spectral graph theory reinforces this interpretation. The second eigenvalue of the graph Laplacian measures overall connectivity. As this spectral gap shrinks toward zero, the graph approaches fragmentation. Fisher information tracks exactly this sensitivity, providing a scalar measure of how close the system is to topological phase transitions.

\subsection{Granger Causality and Effective Edges}

The multiway causal graph potentially contains an exponentially large number of edges representing all possible transitions between configurations. However, not all edges represent genuine causal relationships in the predictive sense. Granger causality provides a principled criterion for identifying which edges carry essential causal information.

An edge from configuration $\Sigma_i$ to $\Sigma_j$ represents a Granger-causal relationship if knowing the system passed through $\Sigma_i$ improves prediction of reaching $\Sigma_j$ beyond what other available information already provides. Edges that fail this test represent spurious correlations or coincidental co-occurrences rather than genuine causal pathways.

This criterion enables construction of an effective causal graph that preserves essential dynamics while removing non-predictive clutter. In practice, edges along the gradient of the action functional typically satisfy Granger causality because systems actively move in these directions through optimization. Edges that merely reflect possible but rarely realized transitions often fail the test and can be pruned.

The effective causal graph resulting from this pruning manages computational complexity without arbitrary approximation. Rather than imposing ad hoc limits on branching factor, we eliminate edges that contribute to apparent complexity without carrying causal information. The resulting graph remains computationally tractable while preserving the causal structure essential for risk assessment.

\subsection{Emergence and Structure-Function Coupling}

A final dimension of fragility concerns the relationship between structural changes and functional outcomes. Systems differ in how tightly function couples to structure. Low emergence or high hardness means specific configurations uniquely determine specific functional outcomes. Any structural damage directly translates to functional loss. High emergence or low hardness means the same function can arise from different structures through compensatory mechanisms.

In the multiway graph, low emergence appears as a one-to-one mapping between structural configurations and functional equivalence classes. Each node represents a unique combination of structure and function. Loss of any particular configuration eliminates the associated function. High emergence appears as many-to-one mapping where multiple structurally distinct configurations occupy the same functional equivalence class. They achieve the same outcome through different mechanisms.

This connects directly to the degeneracy concept. Degenerate systems exhibit high emergence because multiple structural arrangements generate the same functional output. This provides robustness through what might be called functional redundancy without structural redundancy. The system maintains function despite structural perturbations by shifting between configurations in the same functional equivalence class.

Consider ecosystem succession represented in a multiway graph where nodes are spatial vegetation patterns. Low emergence would mean a specific spatial arrangement uniquely determines productivity. A grassland mosaic at particular locations generates specific carbon sequestration rates, and no other arrangement can achieve this. High emergence means wildly different spatial patterns can generate identical productivity through compensatory mechanisms. Grassland mosaic and shrubland matrix, despite complete structural differences, achieve the same carbon sequestration through different photosynthetic strategies and root architectures.

The multiway graph makes this visible by grouping nodes into functional equivalence classes. Systems with dense within-class connectivity and sparse between-class connectivity exhibit low emergence. Structural perturbations tend to move the system out of its functional class. Systems with rich pathways allowing navigation within functional classes while maintaining function despite structural changes exhibit high emergence.

\section{Implications and Applications}

This framework transforms how we think about risk across domains by revealing geometric and topological properties invisible to classical approaches.

\subsection{Climate Adaptation as Pathway Topology}

Climate storylines represent alternative evolution paths of the climate system. Each storyline is a complete narrative from initial conditions through circulation patterns to regional impacts and adaptation responses. The multiway graph representation makes explicit what verbal storylines leave implicit: the branching structure of possible climate futures and how decisions at each node constrain or expand subsequent options.

A climate configuration includes temperature and precipitation fields everywhere on Earth, ocean heat content at all depths, sea ice extent, circulation patterns like AMOC strength, and atmospheric states like ENSO phase. The action functional balances adaptation investment against avoided damages plus penalties for crossing tipping points like permafrost collapse or ice sheet loss.

The framework reveals that two pathways with identical global average temperature might pass through fundamentally different configurations. One maintains strong ocean circulation while the other exhibits AMOC collapse. These are distinct nodes in configuration space with completely different regional impact patterns and adaptation requirements. Classical risk assessment treating "2°C warming" as a single scenario misses this essential heterogeneity.

Pathway topology analysis shows whether climate futures exhibit genuine degeneracy with multiple independent adaptation strategies or mere redundancy where apparent options share common vulnerabilities. Fisher information detects when the climate system approaches tipping points by measuring topological sensitivity. As critical thresholds near, small changes in emissions or adaptation timing cause dramatic multiway graph reorganization, signaling approaching bifurcations.

\subsection{Infrastructure Resilience as Graph Connectivity}

Infrastructure networks evolve through investment decisions that add components, upgrade capacities, or allow aging and degradation. Each decision transforms the network from one configuration to another, creating a trajectory through the multiway graph where nodes are complete network states including topology, capacities, flows, and component ages.

The action functional balances construction and congestion costs against throughput utility plus penalties for capacity violations or component failures. Optimization concentrates probability on investment sequences that achieve target service levels at minimum cost while avoiding overload cascades.

Pathway analysis distinguishes between networks with genuine topological diversity and those with apparent but brittle redundancy. A network with three parallel routes might seem robust, but if all routes share common vulnerabilities like geographic co-location or dependence on shared upstream infrastructure, they represent redundancy rather than degeneracy. The multiway graph makes this visible by showing that these pathways remain close in configuration space.

Fisher information provides early warning of approaching percolation thresholds where minor disruptions trigger network fragmentation. As the network operates near capacity limits or aging components approach failure, small perturbations cause disproportionate impacts. The multiway graph topology becomes highly sensitive, with slight changes in component status or demand patterns causing large reorganization of viable flow configurations.

\subsection{Ecosystem Dynamics as Alternative Stable States}

Ecosystems following disturbance can progress along multiple successional pathways toward different stable states \cite{holling1973resilience}. Each pathway represents a trajectory through configuration space where nodes are complete spatial vegetation patterns including species composition, biomass distribution, soil properties, and connectivity patterns.

The action functional balances disturbance response costs against productivity utility plus penalties for biodiversity loss or invasive species dominance. Different management interventions steer the system along different trajectories by altering relative action costs of various transitions.

The multiway graph makes alternative stable states explicit. Disconnected components of the graph represent configurations with no feasible transitions between them, corresponding to alternative stable states separated by ecological barriers. The graph topology reveals whether multiple pathways exist to desired states or whether early decisions irreversibly commit the system to particular outcomes.

Degeneracy appears as multiple structurally different configurations achieving similar ecosystem function. A grassland mosaic and shrubland matrix might generate equivalent carbon storage, water cycling, and habitat value through different mechanisms. This functional equivalence despite structural differences provides adaptive capacity. If one pathway becomes blocked, alternatives exist that reach the same functional target through different routes.

\subsection{Universal Patterns}

Across domains, the framework reveals common geometric patterns. Fragile systems exhibit few pathways with high redundancy and structural coupling. The multiway graph is sparse with bottlenecks, and pathways that exist share failure modes. Fisher information is high, indicating topological instability. Emergence is low, with tight structure-function coupling.

Robust systems exhibit many pathways with genuine degeneracy and structural independence. The multiway graph is richly connected with diverse branching. Pathways diverge in configuration space and fail independently. Fisher information is low, indicating topological stability. Emergence is high, with multiple structural solutions achieving equivalent function.

Antifragile systems gain options under stress. Disturbances create new branching in the multiway graph rather than pruning existing branches. The topology becomes richer rather than sparser under perturbation. This appears as regions of the graph where stress opens previously inaccessible transitions, connecting formerly isolated components or creating new pathways through configuration space.

\section{Toward a New Epistemic Foundation}

This framework establishes a new way of thinking about risk that moves beyond static factor multiplication toward dynamic pathway analysis. Risk is not a scalar property of isolated systems but a geometric property of trajectory distributions shaped by optimization under constraints.

The key epistemic shifts include recognizing that system states are complete configurations with internal structure rather than simple points in abstract state space. This structure matters for risk because it determines failure modes, adaptation options, and response mechanisms. Understanding that optimization reshapes accessible futures means we cannot treat all scenarios as equally probable. Systems actively navigate toward low-action paths, concentrating probability in ways that traditional scenario planning misses.

Distinguishing redundancy from degeneracy reveals that apparent pathway diversity may mask structural coupling and shared vulnerabilities. Counting options is insufficient without analyzing their independence. Measuring topological sensitivity through Fisher information provides early warnings that go beyond monitoring individual variables. Approaching tipping points reveal themselves through increasing sensitivity of graph structure to perturbations.

The multiway causal graph representation makes abstract trajectory distributions concrete and computable. Stakeholders can visualize branching futures, identify critical decisions that constrain subsequent options, and recognize where apparent robustness masks latent fragility. This visual and computational tractability transforms the framework from theoretical construct to practical methodology.

The framework does not predict which specific trajectory will be realized. This is impossible in non-ergodic systems with irreducible uncertainty. Instead, it characterizes the topology of accessible futures: how many viable pathways exist, how independent they are, how sensitive the structure is to perturbations, and how tightly function couples to structure. This represents a shift from false precision about unknowable futures toward rigorous characterization of possibility spaces.

Classical risk science asks "what is the probability of event X?" and provides point estimates with confidence intervals. Path-distributional risk science asks "what is the topology of pathway space?" and provides geometric and topological characterizations. This is not merely technical refinement but epistemological reorientation. Risk becomes a property of how systems navigate branching possibility spaces rather than a property of isolated events in isolation.

This new epistemic foundation opens research directions spanning theory, computation, and application. Theoretical work might develop topology-preserving coarse-graining methods that simplify multiway graphs while retaining essential structure, extend the framework to systems with continuous state spaces and infinite-dimensional configurations, or connect pathway topology to other complexity measures like computational irreducibility and algorithmic information.

Computational research might develop efficient algorithms for computing topological invariants on large multiway graphs, create visualization methods that effectively communicate pathway structure to decision makers, or build machine learning approaches that discover configuration representations optimized for revealing risk-relevant topology.

Applied research might validate the framework against historical case studies where systems underwent anticipated or unanticipated transitions, develop domain-specific action functionals that capture optimization trade-offs in particular contexts, or create decision support tools that help stakeholders navigate multiway graphs to identify robust strategies.

The framework suggests new approaches to intervention and adaptation. Rather than optimizing for single predicted futures, we design for pathway diversity and structural independence. Rather than assuming robustness from option counts, we assess coupling and sensitivity. Rather than waiting for systems to approach critical points, we monitor topological early warnings. Rather than treating systems as static objects to be protected, we recognize them as dynamic processes to be steered through configuration space toward regions with favorable topology.

This perspective applies beyond traditional risk domains. Technological innovation follows branching pathways through design space with path dependence and lock-in effects visible as multiway graph topology. Social-ecological systems evolve through coupled configurations of institutions, ecosystems, and infrastructure with emergent properties arising from structure-function relationships. Even scientific progress might be understood as navigating a multiway graph of possible theoretical frameworks where current paradigms constrain but do not determine accessible futures.

The epistemic foundation we propose treats risk as emerging from pathway geometry rather than scenario probability. This shift from probabilistic to topological thinking, from scalar estimates to distributional structure, from static snapshots to dynamic evolution, establishes a new basis for understanding how complex systems navigate uncertainty. The multiway causal graph provides both conceptual clarity and computational tractability, making this abstract framework concrete and applicable.

Risk science has long sought to move beyond the limitations of hazard-exposure-vulnerability multiplication while retaining rigor and quantification. The path-distributional framework with its multiway graph representation offers such an alternative by grounding risk assessment in the geometry of branching causal pathways. This is not the final word but a foundation upon which richer theory and more sophisticated applications can build. The essential insight that risk arises from pathway topology shaped by optimization provides an organizing principle for understanding adaptive capacity, detecting approaching transitions, and distinguishing genuine from illusory robustness across domains.


\bibliographystyle{plain}
\begin{thebibliography}{99}

\bibitem{kaplan1981quantitative}
Stanley Kaplan and B. John Garrick.
On the quantitative definition of risk.
\textit{Risk Analysis}, 1(1):11--27, 1981.

\bibitem{shannon1948mathematical}
Claude E. Shannon.
A mathematical theory of communication.
\textit{The Bell System Technical Journal}, 27(3):379--423, 1948.

\bibitem{holling1973resilience}
Crawford S. Holling.
Resilience and stability of ecological systems.
\textit{Annual Review of Ecology and Systematics}, 4(1):1--23, 1973.

\bibitem{mcphearson2016scientists}
Timon McPhearson, Steward T. A. Pickett, et al.
Scientists must have a say in the future of cities.
\textit{BioScience}, 66(3):175--182, 2016.

\bibitem{markolf2018interdependent}
Samuel A. Markolf, Mikhail V. Chester, et al.
Interdependent infrastructure as linked social, ecological, and technological systems (SETSs) to address lock-in and enhance resilience.
\textit{Earth's Future}, 6(12):1638--1659, 2018.

\bibitem{krueger2019requisite}
Elisabetta Krueger et al.
Requisite resilience: Aligning internal variety with environmental variety in complex adaptive systems.
\textit{Urban Trends Working Papers}, 2019.

\bibitem{wolfram2020class}
Stephen Wolfram.
A Class of Models with the Potential to Represent Fundamental Physics.
\textit{Wolfram Media}, 2020.

\bibitem{bak1987self}
Per Bak, Chao Tang, and Kurt Wiesenfeld.
Self-organized criticality: An explanation of 1/f noise.
\textit{Physical Review Letters}, 59(4):381, 1987.

\bibitem{friston2010free}
Karl Friston.
The free-energy principle: a unified brain theory?
\textit{Nature Reviews Neuroscience}, 11(2):127--138, 2010.

\bibitem{feynman1965quantum}
Richard P Feynman and Albert R Hibbs.
\textit{Quantum Mechanics and Path Integrals}.
McGraw-Hill, New York, 1965.

\bibitem{shepherd2018storylines}
Theodore G Shepherd et al.
Storylines: an alternative approach to representing uncertainty in physical aspects of climate change.
\textit{Climatic Change}, 151(3-4):555--571, 2018.

\bibitem{wolfram2002}
Stephen Wolfram.
\textit{A New Kind of Science}.
Wolfram Media, 2002.

\bibitem{granger1969}
Clive WJ Granger.
Investigating causal relations by econometric models and cross-spectral methods.
\textit{Econometrica}, 37(3):424--438, 1969.

\bibitem{tononi1999measures}
Giulio Tononi, Olaf Sporns, and Gerald M Edelman.
Measures of degeneracy and redundancy in biological networks.
\textit{Proceedings of the National Academy of Sciences}, 96(6):3257--3262, 1999.

\bibitem{edelman2001degeneracy}
Gerald M Edelman and Joseph A Gally.
Degeneracy and complexity in biological systems.
\textit{Proceedings of the National Academy of Sciences}, 98(24):13763--13768, 2001.

\bibitem{scheffer2009early}
Marten Scheffer et al.
Early-warning signals for critical transitions.
\textit{Nature}, 461(7260):53--59, 2009.

\bibitem{newman2010}
Mark Newman.
\textit{Networks: An Introduction}.
Oxford University Press, 2010.

\end{thebibliography}
\end{document}