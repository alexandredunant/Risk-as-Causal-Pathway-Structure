\documentclass[12pt]{article}
\usepackage[utf8]{inputenc}
\usepackage[english]{babel}
\usepackage{amsmath}
\usepackage{amssymb}
\usepackage{graphicx}
\usepackage{geometry}
\usepackage{hyperref}
\usepackage{setspace}
\usepackage{float}
\usepackage{natbib}
\geometry{margin=1in}
\setstretch{1.2}

\title{Risk as Causal Pathway Structure: A Multiway Topological Framework for Systemic Futures}
\author{}
\date{}

\begin{document}
\maketitle

\begin{abstract}
Risk analysis traditionally treats risk as a static quantity derived from hazard, exposure, and vulnerability. This framing fails to capture the deep uncertainty and branching futures of complex Social--Ecological--Technological Systems (SETS). We propose a generative, rule-based framework in which risk emerges not from static probabilities but from the topology of the multiway causal graph---the ensemble of all possible futures generated by local transformation rules. We introduce topological metrics such as pathway multiplicity, reconvergence coefficient, topological sensitivity, and algebraic connectivity to quantify resilience and fragility. Through a synthetic comparison of a degradation-dominant regime (``Glass'') and an adaptive regime (``Plant''), we demonstrate how resilience corresponds to structural degeneracy and recombination cycles while fragility corresponds to narrowing connectivity. This framework provides a new epistemic basis for linking causal structure to decision-relevant scenario diversity, offering principled metrics for robust decision-making under deep uncertainty.
\end{abstract}

\section{Introduction}

The classical triplet formulation of risk (\emph{hazard} $\times$ \emph{exposure} $\times$ \emph{vulnerability}) \citep{kaplan1981quantitative, turner2003framework} provides useful static intuition but cannot represent the branching, path-dependent futures characteristic of complex Social--Ecological--Technological Systems (SETS). Real systems evolve through local interactions, adaptive decisions, and structural feedbacks that exhibit emergent behavior and nonlinear dynamics \citep{nicolis1977self, mitchell2009complexity}. Under conditions of deep uncertainty \citep{walker2013deep, lempert2003shaping}, risk cannot be adequately captured as a scalar probability-weighted outcome but must instead be understood as a geometrical property of the space of possible futures.

We propose a fundamental shift from static assessment to a \emph{generative} viewpoint, wherein risk emerges as a property of the \emph{multiway causal graph} generated by the iterative application of local transformation rules. This perspective emphasizes the structure, connectivity, and convergence properties of possible futures, enabling risk evaluation through topological analysis rather than static aggregate measures. Drawing on rewriting systems \citep{wolfram2002}, network science \citep{buldyrev2010catastrophic}, resilience theory \citep{holling1973resilience, scheffer2009critical}, storyline approaches to scenario design \citep{shepherd2018storylines, sillmann2021event}, and insights about path dependence and lock-in \citep{arthur1989competing, david1985clio}, our framework unifies these conceptual threads into a coherent, operationalizable formalism for systemic risk analysis.

The framework operates at multiple scales depending on analytical goals and data availability. At the conceptual level, it reformulates risk as an emergent property of causal topology rather than as a static aggregate of component vulnerabilities. At the formal level, it provides a precise mathematical language for describing how systems transition between configurations and how causal pathways interact, converge, or diverge. At the computational level, when transformation rules are empirically calibrated to specific systems, the framework becomes a predictive simulation tool capable of generating concrete scenario ensembles for decision support.

\section{Formalism: States, Rules, and the Multiway Graph}

\subsection{Configuration Space and Transformation Rules}

We represent a system state as a graph configuration \(\Sigma = (V, E)\), where \(V\) denotes the set of components spanning social, ecological, and technological domains, and \(E\) represents the couplings, dependencies, and interactions between these components \citep{markolf2018interdependent}. This representation recognizes that SETS are fundamentally relational, with system behavior emerging from the pattern of connections as much as from the properties of individual elements. The graph representation naturally accommodates heterogeneous system components, explicitly represents cascading pathways central to systemic risk \citep{buldyrev2010catastrophic}, and provides intuitive visualization of system structure and evolution.

System evolution is driven by a set of local transformation rules \(R = \{r_1, \ldots, r_n\}\), each of which acts as a rewriting operator on graph configurations. At their most fundamental level, these rules are purely structural operations that manipulate the graph topology through node and edge creation, deletion, or modification. Formally, each rule maps a configuration to a set of possible successor configurations:
\[
    r: \Sigma \rightarrow \{\Sigma'_1, \ldots, \Sigma'_k\}.
\]
The multiplicity of successors (\(k > 1\)) reflects genuine uncertainty about which outcome will occur when the rule is applied, capturing both aleatory uncertainty (inherent stochasticity) and epistemic uncertainty (limited knowledge about system response).

Crucially, the transformation rules themselves are defined solely by their graph rewriting operations—they specify which nodes or edges are added, removed, or modified under what conditions. Once a rule is applied to a configuration, we can observe and quantify its effects through a signature vector:
\[
\delta \vec{r} = (\Delta U, \Delta E, \Delta C),
\]
where the components represent changes in utility or performance (\(\Delta U\)), energetic, economic, or operational effort (\(\Delta E\)), and constraints on future evolution (\(\Delta C\)). These quantities are not determinants of the rule but rather emergent properties that can be measured once the rule is applied to a specific configuration. The same graph operation may produce different observable signatures depending on the configuration to which it is applied and the context-specific interpretation of utility, effort, and constraint.

The constraint component \(\Delta C\) measures how a rule alters the \emph{space of possible futures}, representing a crucial but often overlooked dimension of system evolution. Positive values of \(\Delta C\) indicate increased restriction of future options, corresponding to loss of adaptive capacity or path-narrowing lock-in \citep{arthur1989competing}. Negative values indicate increased flexibility, corresponding to creation of new options or expanded adaptive capacity. This notion connects to entropy and accessible phase space in statistical mechanics \citep{jaynes1957information, jaynes1957information2} and to functional diversity in ecology \citep{oliver2015biodiversity}. By explicitly tracking constraints, our framework captures path dependence and hysteresis effects that are central to understanding systemic risk but difficult to represent in conventional risk assessment.

\begin{table}[H]
\centering
\begin{tabular}{l l c c c}
\hline
Domain & Rule & $\Delta U$ & $\Delta E$ & $\Delta C$ \\
\hline
Infrastructure & Pipe rupture (edge deletion) & $\downarrow$ & $\downarrow$ & $\uparrow$ \\
Infrastructure & Parallel capacity (edge creation) & $\uparrow$ & $\uparrow$ & $\downarrow$ \\
Ecological & Extinction (node deletion) & $\downarrow$ & $\downarrow$ & $\uparrow$ \\
Ecological & Mutualism (edge creation) & $\uparrow$ & $\uparrow$ & $\downarrow$ \\
Social & Tie dissolution (edge deletion) & $\downarrow$ & $\downarrow$ & $\uparrow$ \\
Social & Bridging (edge creation) & $\uparrow$ & $\uparrow$ & $\downarrow$ \\
\hline
\end{tabular}
\caption{Examples of how graph rewriting operations map to observable signatures across SETS domains. The fundamental operations—node and edge creation or deletion—produce measurable consequences in utility, effort, and constraint that can be interpreted in domain-specific contexts.}
\end{table}

\subsection{The Multiway Causal Graph}

Starting from an initial configuration \(\Sigma_0\), recursive application of the rule set \(R\) generates the multiway causal graph \(G_{\mathrm{MW}}\), where nodes represent distinct system configurations and edges represent specific rule applications. This graph encodes the entire accessible future state space, capturing not only individual trajectories but also the relationships between different pathways, including their convergence, divergence, and potential reconnection.

The multiway structure generalizes conventional scenario trees by explicitly representing causal relationships between branches. In a simple tree, once pathways diverge they never reconnect, implying that early choices irrevocably determine long-term outcomes. In the multiway graph, pathways can converge back to common configurations through different sequences of transformations, reflecting the possibility that distinct causal sequences may lead to functionally equivalent states. This convergence property, which we term \emph{causal closure}, is central to understanding system resilience. The concept of multiway evolution originates in the study of formal rewriting systems \citep{wolfram2002}. We adapt this framework to SETS by interpreting nodes as socio-ecological-technological configurations and edges as interventions, adaptations, or disturbances that transform system structure.

\subsection{Topological Metrics for Risk Quantification}

We propose three complementary metrics based on information theory, statistical physics, and spectral graph theory that characterize the topological structure of the possibility space.

\subsubsection{Pathway Multiplicity and Structural Degeneracy}

The diversity of viable pathways through the multiway graph provides a direct measure of system flexibility and adaptive capacity. We quantify this through two complementary topological measures. First, the raw pathway count \(N_{\mathrm{paths}}\) enumerates the number of distinct directed paths from the initial configuration to configurations at a specified time horizon or satisfying terminal criteria. Second, the reconvergence coefficient \(\rho\) measures structural degeneracy by quantifying the frequency of pathway mergers:
\[
\rho = \frac{\text{number of nodes with } \geq 2 \text{ incoming edges from distinct paths}}{\text{total number of nodes}}.
\]
High \(N_{\mathrm{paths}}\) indicates a rich, diverse possibility space with many structurally distinct futures, suggesting resilience through multiplicity of options. High \(\rho\) indicates that different causal sequences frequently reunite into common configurations, reflecting adaptive capacity through multiple routes to functional states. Low values of both metrics indicate narrowing of possibilities and lack of redundancy, suggesting vulnerability to perturbations that block the limited available pathways.

\subsubsection{Topological Sensitivity to Forcing}

While pathway multiplicity measures the current diversity of the possibility space, topological sensitivity quantifies how this structure changes under variations in system parameters or forcing conditions. For a parameter \(\theta\) that affects rule application (such as disturbance intensity, resource availability, or intervention capacity), we measure sensitivity through the rate of change of topological properties:
\[
S(\theta) = \left| \frac{\partial N_{\mathrm{paths}}(\theta)}{\partial \theta} \right| + \left| \frac{\partial \lambda_2(\theta)}{\partial \theta} \right| + \left| \frac{\partial \rho(\theta)}{\partial \theta} \right|,
\]
where each term captures how pathway count, algebraic connectivity, or reconvergence frequency changes with \(\theta\). High \(S(\theta)\) indicates that small changes in conditions substantially alter the topology of the possibility space, signifying a system near a structural transition where the multiway graph undergoes qualitative reorganization. Regions where \(S(\theta)\) spikes correspond to critical transition points where the system crosses from one topological regime to another, providing early warning of impending structural changes in the possibility space.

\subsubsection{Spectral Gap (Structural Connectivity)}

The algebraic connectivity of the multiway graph characterizes how well-connected different regions of the possibility space are. We quantify this through the spectral gap, defined as the second-smallest eigenvalue \(\lambda_2\) of the graph Laplacian matrix \(L\). High \(\lambda_2\) indicates strong connectivity, meaning that many alternative pathways exist between different system states, facilitating adaptation and recovery from perturbations. Low \(\lambda_2\) indicates fragile connectivity, where the possibility space consists of poorly connected regions. The limiting case \(\lambda_2 \to 0\) corresponds to complete disconnection, where the multiway graph fragments into isolated components representing distinct regimes between which no transitions are possible. This topological collapse signifies catastrophic loss of adaptive capacity.

\subsection{Linking to Storyline-Based Scenario Design}

The storyline approach to scenario design \citep{shepherd2018storylines, sillmann2021event} constructs physically consistent narratives that explore how different combinations of factors lead to distinct outcomes, emphasizing causal coherence over probabilistic weighting. Our topological metrics map directly onto key concerns in storyline-based analysis. The pathway count \(N_{\mathrm{paths}}\) quantifies the number of structurally distinct storylines that merit consideration. The algebraic connectivity \(\lambda_2\) measures the feasibility of transitions between storylines, indicating whether they represent distinct regimes separated by high barriers or closely related possibilities. The reconvergence coefficient \(\rho\) reveals whether different storylines can reunite into common functional states, indicating robustness of outcomes to causal details. Topological sensitivity \(S(\theta)\) identifies which parameters or interventions most substantially alter the structure of possibilities, guiding both scientific investigation and policy design by revealing leverage points where interventions can shift the system toward more favorable regimes.

\section{Case Study: Glass and Plant Regimes}

To illustrate how different rule structures generate distinct topological regimes, we compare two archetypal systems that differ fundamentally in their balance of degradation and repair mechanisms.

The \textbf{Glass} regime is dominated by deletion rules representing degradation, damage, or failure processes without compensating repair or regeneration mechanisms. Rules progressively remove edges and nodes from the system graph, reducing connectivity and eliminating functional components. This regime corresponds to brittle infrastructure lacking maintenance, ecosystems experiencing extinction cascades without compensating speciation, or social networks undergoing dissolution without reformation.

The \textbf{Plant} regime maintains a balance between deletion and creation rules, with local repair mechanisms that can restore lost connections through transitivity principles. When edges are deleted, nearby nodes can form compensating connections, maintaining overall connectivity through structural reorganization. This regime corresponds to maintained infrastructure with redundancy, ecosystems with functional compensation and niche filling, or social networks with bridge-building and tie reformation.

These archetypal labels capture essential differences in adaptive capacity. Glass shatters irreversibly when struck, with cracks propagating without self-repair. Plants heal damage through local growth and structural reorganization, with pruned branches stimulating compensatory sprouting.

\subsection{Topological Divergence and Quantification}

Despite identical initial configurations and exposure to the same disturbances, the Glass and Plant regimes evolve dramatically different multiway graph structures. The Glass system exhibits a collapsing partial mesh topology where diverse initial pathways progressively converge to a unique sink state representing complete system failure. Early in the evolution, the multiway graph shows apparent diversity as different components fail in different sequences. However, this diversity is illusory because all pathways ultimately lead to the same degenerate outcome. The pathway count \(N_{\mathrm{paths}}\) declines monotonically as the system ages, the reconvergence coefficient \(\rho\) approaches unity (trivial convergence to the same failure state), and the spectral gap \(\lambda_2\) approaches zero, signifying topological collapse.

\begin{figure}[H]
    \centering
    \includegraphics[width=\textwidth]{multiway_futures_comparison.png}
    \caption{Comparative evolution of the Glass (left) and Plant (right) multiway causal graphs over equivalent time horizons. The Glass system exhibits a collapsing topology where all paths converge to a single failure state, showing high early diversity that progressively narrows to a unique terminal configuration. The Plant system exhibits a rich, recombining topology with high degeneracy, maintaining multiple viable configurations through repair cycles that prevent convergence to failure. Despite identical initial hazard and exposure conditions, the difference in transformation rules produces radically different possibility spaces, demonstrating that risk emerges from causal topology rather than from static vulnerability metrics.}
    \label{fig:comparison}
\end{figure}

In contrast, the Plant system exhibits rich recombination cycles that produce structural degeneracy, where multiple distinct transformation sequences lead to functionally equivalent configurations. The multiway graph develops a highly connected mesh structure with numerous reconvergence points, maintaining high \(N_{\mathrm{paths}}\), high \(\rho\), and high \(\lambda_2\) throughout evolution. Even as individual pathways diverge due to stochastic disturbances, repair mechanisms enable the system to explore and reconnect different regions of the configuration space.

Figure 1 visualizes this fundamental topological divergence. The contrasting structures demonstrate that risk is not determined by initial conditions or disturbance intensity alone but emerges from the topology of transformation rules. Systems exposed to identical hazards can exhibit radically different risk profiles depending on their capacity for adaptive reorganization.

\begin{figure}[H]
    \centering
    \includegraphics[width=0.8\textwidth]{topological_risk_table.png}
    \caption{Quantitative topological metrics comparing the Glass and Plant regimes. The Glass system shows monotonic decline in pathway count (\(N_{\mathrm{paths}}\)), trivial reconvergence to a single failure state (high \(\rho\) but to a catastrophic endpoint), high topological sensitivity \(S(\theta)\) near collapse, and vanishing spectral gap (\(\lambda_2 \to 0\)) signifying loss of connectivity. The Plant system maintains high pathway diversity, nontrivial reconvergence to functional states, moderate sensitivity, and strong connectivity throughout evolution. These metrics quantify the intuitive distinction between brittle and resilient systems in terms of precise features of the multiway causal graph.}
    \label{fig:metrics}
\end{figure}

Figure 2 quantifies these topological differences through the metrics introduced in Section 2.3. The stark contrasts in \(N_{\mathrm{eff}}\), \(I(\theta)\), and \(\lambda_2\) provide objective, quantifiable measures of resilience and fragility that enable rigorous comparison of alternative system designs.

\subsection{Causal Invariance and Closure}

A crucial topological property distinguishing the two regimes is the degree of \emph{causal invariance}---the extent to which different sequences of transformations applied to the same initial state lead to equivalent final configurations. In the Glass regime, causal invariance is trivially satisfied through convergence: all pathways eventually reach the same absorbing failure state regardless of the sequence of edge deletions. However, this represents the worst possible form of invariance, where insensitivity to pathway details arises from collapse of all diversity rather than from adaptive compensation. This trivial closure offers no resilience benefit because the unique endpoint is catastrophic.

The Plant regime exhibits nontrivial causal invariance through functional equivalence classes. Multiple distinct transformation sequences reunite into configurations that are structurally different but functionally equivalent, providing the same performance and adaptive capacity. For example, after suffering edge deletions in different locations, the system may form compensating connections through different mechanisms, arriving at distinct graph configurations that nonetheless support equivalent function. This nontrivial closure provides genuine resilience because the reconvergent states maintain viability rather than representing shared failure. The concept connects to fundamental ideas in physics, where symmetries and conservation laws arise from invariance principles \citep{landau1976mechanics, arnold1989mathematical}. In our context, repair rules act as symmetry operations that restore certain structural properties despite local perturbations.

\subsection{Non-Closure and Lock-In}

Systems lacking adequate repair mechanisms exhibit non-closure, where transformation order fundamentally matters because different sequences lead to irreversibly distinct outcomes. Formally, the transformation rules fail to commute:
\[
R_a(R_b(\Sigma)) \neq R_b(R_a(\Sigma)),
\]
meaning that applying rule \(R_a\) followed by rule \(R_b\) produces a different configuration than applying them in reverse order. This non-commutativity induces hysteresis and path dependence, where early choices constrain later possibilities through irreversible commitment.

In the Glass regime, edge deletion is inherently non-commutative with respect to any potential repair rules (which are absent). Once an edge is deleted, any functionality depending on that connection is permanently lost, closing off entire branches of the possibility space. The system evolves as a pure divergence tree where pathways never reconnect, and the configuration space progressively fragments into isolated, unreachable regions. This loss of connectivity manifests as lock-in effects \citep{arthur1989competing, david1985clio} familiar from economic systems with increasing returns, technological systems with compatibility requirements, and ecological systems approaching extinction thresholds. Once locked into a degraded state, the system cannot access previously available configurations even if conditions change.

The Plant regime avoids lock-in through repair rules that provide partial reversibility and alternative pathways to functionally equivalent states. While specific transformation sequences remain path-dependent in their details, the functional consequences exhibit robustness to ordering variations. This flexibility emerges from structural degeneracy, where multiple configurations can satisfy the same functional requirements, combined with repair mechanisms that enable navigation between these functionally equivalent but structurally distinct states.

\section{Implications for Risk Assessment Practice}

The Glass-Plant comparison demonstrates how identical initial conditions and disturbance exposures can produce radically different risk profiles depending on the topology of transformation rules. This finding underscores the limitations of conventional risk assessment approaches that focus on hazard and exposure while treating vulnerability as a static property. Our analysis reveals that system evolution is fundamentally shaped by the balance between degradation and repair processes, and that resilience emerges from the structural properties of the rule set rather than from any single component or parameter.

The topological metrics introduced here translate abstract resilience concepts into quantifiable features of the possibility space that can guide policy design, infrastructure planning, and ecosystem management. Systems with declining \(N_{\mathrm{paths}}\) and \(\lambda_2\) signal deteriorating adaptive capacity even before functional performance degrades, enabling proactive intervention. Conversely, systems maintaining high pathway diversity and connectivity demonstrate robustness to perturbations. Spikes in topological sensitivity \(S(\theta)\) provide early warning of critical transitions, indicating when the system approaches thresholds where small changes in forcing or parameters could trigger regime shifts in the structure of the possibility space.

The framework also provides practical guidance for intervention design. Policies that preserve or create structural degeneracy enhance resilience by maintaining multiple pathways to desired outcomes. Interventions that reduce \(\Delta C\) expand future options, while those that increase \(\Delta C\) may offer short-term benefits at the cost of long-term flexibility. By explicitly representing the space of possible futures rather than aggregating over scenarios, the multiway approach enables robust decision-making that accounts for deep uncertainty without requiring precise probability assignments.

\section{Conclusion}

This work proposes a fundamental reconceptualization of systemic risk as a property of the topology of possible futures rather than as a static scalar derived from component vulnerabilities. The multiway causal graph framework provides a generative mathematical structure in which resilience corresponds to structural degeneracy and adaptive reconvergence, while fragility corresponds to narrowing connectivity and irreversible lock-in. This shift from static to dynamic, from scalar to topological, and from aggregate to causal enables new forms of reasoning about systemic risk under deep uncertainty.

The topological metrics introduced—pathway multiplicity, reconvergence coefficient, topological sensitivity, and algebraic connectivity—provide decision-relevant indicators aligned with emerging approaches such as storylines \citep{shepherd2018storylines, sillmann2021event}. These metrics translate abstract resilience concepts into quantifiable features of the possibility space that can guide robust decisions under deep uncertainty \citep{walker2013deep, lempert2003shaping}.

The multiway causal graph perspective offers not merely a technical tool but a conceptual lens through which to understand how systems navigate uncertain futures. By representing risk as emerging from the structure of possibilities rather than from static properties, this framework aligns with growing recognition that deep uncertainty demands approaches emphasizing robust adaptation over precise prediction \citep{lempert2003shaping, walker2013deep}. Empirical rule calibration and stakeholder engagement remain important next steps for translating this conceptual framework into operational decision support tools. As humanity confronts increasingly complex, interconnected challenges from climate change \citep{lenton2008tipping} to technological disruption to social transformation, frameworks that capture the branching, path-dependent, and fundamentally uncertain nature of systemic evolution become essential for navigating toward desirable futures.

\bibliographystyle{plainnat}
\bibliography{references}

\end{document}