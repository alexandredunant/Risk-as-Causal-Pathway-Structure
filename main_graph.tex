\documentclass[11pt,a4paper]{article}

% ---------------------------------------------------------
% PACKAGES
% ---------------------------------------------------------
\usepackage[utf8]{inputenc}
\usepackage[T1]{fontenc}
\usepackage{geometry}
\usepackage{natbib}
\usepackage{graphicx}
\usepackage{amsmath, amssymb}
\usepackage{booktabs}
\usepackage{float}
\usepackage{hyperref}
\usepackage{array}
\usepackage{xcolor}

\geometry{margin=1in}

% ---------------------------------------------------------
% TITLE
% ---------------------------------------------------------
\title{A Rule-Based Multiway Framework for Risk, Fragility, and Resilience:\\
Utility--Effort--Constraints in Unweighted Graph Dynamics}

\author{Your Name}
\date{\today}

% ---------------------------------------------------------
% DOCUMENT
% ---------------------------------------------------------
\begin{document}
\maketitle

\begin{abstract}
We introduce a discrete multiway framework for analysing fragility and resilience in systems represented as unweighted graphs. 
Instead of relying on a continuous action functional to select or optimise trajectories, we define microscopic generative rules that modify graph topology according to changes in three fundamental dimensions: \textbf{Utility}, \textbf{Effort}, and \textbf{Constraints}. 
All possible transformations are retained, producing a multiway causal graph that captures the full space of reachable future configurations.
The classical action functional emerges as a macroscopic summary of rule applications, not as a selector of optimal paths.
Using two contrasting systems---a \emph{glass} species (pure degradation) and a \emph{plant} species (degradation + repair)---we demonstrate how resilience emerges from rule diversity and manifests in the topology of the multiway state space.
\end{abstract}

% =========================================================
% SECTION 1 — INTRODUCTION
% =========================================================
\section{Introduction}

Many natural and socio-technical systems evolve through discrete structural transformations: they lose, gain, or reorganise connections.
Such systems can be naturally represented as unweighted, undirected graphs whose dynamics are defined not by continuous flows but by structural transitions.
Traditional approaches rely on energy landscapes or action functionals to select optimal or probable trajectories.
However, many systems do not follow optimal paths; they simply evolve according to local rules.

In this work we propose a \textbf{rule-based multiway framework}:
\begin{enumerate}
    \item System states are unweighted graphs.
    \item Microscopic rules encode changes in Utility, Effort, and Constraints.
    \item All rule-generated futures are retained in a multiway causal graph.
    \item Risk emerges from the global topology of this multiway graph.
\end{enumerate}

This perspective treats the action functional not as an optimisation principle but as an emergent macroscopic observable summarising the cumulative effects of rule sequences.

% =========================================================
% SECTION 2 — STATES AS UNWEIGHTED GRAPHS
% =========================================================
\section{States as Unweighted Graph Configurations}

A \emph{state} is defined as an unweighted, undirected graph
\[
S = (V,E),
\]
where $V$ is the fixed set of nodes and $E \subseteq V \times V$ is the set of edges.
Edges have no weights, costs, or capacities.
Thus, all information is encoded in the presence or absence of edges, making the system purely topological.

% =========================================================
% SECTION 3 — RULE-BASED DYNAMICS
% =========================================================
\section{Rule-Based Dynamics: Utility, Effort, Constraints}

We define rules not as abstract transformations but as operations characterised by changes in:
\[
(\Delta U, \Delta E, \Delta C) \in \{\uparrow, \downarrow, \rightarrow\}^3,
\]
representing changes in \textbf{Utility}, \textbf{Effort}, and \textbf{Constraints}.

Each rule acts on the topology of $E$, producing one or more future states.

% ---------------------------------------------------------
% RULE TABLE (UEC)
% ---------------------------------------------------------
\begin{table*}[t]
\centering
\begin{tabular}{p{3cm} p{4cm} p{1.3cm} p{1.3cm} p{1.6cm} p{4cm}}
\toprule
\textbf{Rule Type} & \textbf{Topological Operation} &
\textbf{Utility} & \textbf{Effort} & \textbf{Constraints} &
\textbf{Interpretation} \\
\midrule

Entropy / Stress (edge deletion) &
Remove edge $(a,b)$ &
$\downarrow$ & $\downarrow$ & $\uparrow$ &
Loss of functionality; less effort; more constrained behaviour \\

Local Repair / Adaptation (triangle closure) &
If $(a,b)$ and $(b,c)$ exist, add $(a,c)$ &
$\uparrow$ & $\uparrow$ & $\downarrow$ &
Creates redundancy; increases utility and cost; relaxes structural constraints \\

Node Expansion &
Add node $n$ and edges $(n,i)$ &
$\uparrow$ & $\uparrow$ & $\downarrow$ &
Growth event; increases capabilities; higher effort; greater flexibility \\

Node Pruning &
Remove node $n$ and all adjacent edges &
$\downarrow$ & $\downarrow$ & $\uparrow$ &
Simplification; reduces effort and utility; imposes constraints \\

Rewiring (neutral) &
Replace $(a,b)$ with $(a,c)$ &
$\rightarrow$ & $\rightarrow$ & $\rightarrow$ &
Neutral reorganization without functional change \\

Pattern Reinforcement (motif completion) &
Add missing edge in structural motif &
$\uparrow$ & $\uparrow$ & $\downarrow$ &
Strengthens patterns; adds robustness at increased effort \\

Constraint Suppression &
Remove edges incident to node $n$ &
$\downarrow$ & $\downarrow$ & $\downarrow$ &
Simplifies structure; reduces constraints and capabilities \\

\bottomrule
\end{tabular}
\caption{Rule types expressed through changes in Utility (U), Effort (E), and Constraints (C). Arrows: $\uparrow$ increase, $\downarrow$ decrease, $\rightarrow$ no change.
All rules act on unweighted graph topology.}
\label{tab:UEC_rules}
\end{table*}

% =========================================================
% SECTION 4 — ACTION AS AN EMERGENT QUANTITY
% =========================================================
\section{The Action Functional as an Emergent Observable}

We do not define an external action $A[S]$ that selects or optimises trajectories.  
Instead:
\begin{itemize}
    \item Rules generate all admissible futures.
    \item No trajectory is favoured during construction.
    \item The multiway graph captures the complete state space.
\end{itemize}

The action of a trajectory is a macroscopic summary of the sequence of UEC changes.
For example, one could define:
\[
A(\gamma) = N_{\downarrow U} + N_{\uparrow E} + N_{\uparrow C},
\]
but this is not used to select trajectories.
Action emerges as a descriptive statistic of rule applications, not a prescriptive law.

% =========================================================
% SECTION 5 — MULTIWAY CAUSAL GRAPH
% =========================================================
\section{Multiway Causal Graph}

Starting from an initial state $S_0$, we apply all rules in parallel.
This produces a second layer of states, which in turn produce a third, and so on.

The resulting structure is a directed acyclic graph:
\[
\mathcal{G}_{MW} = (\mathcal{M}, \mathcal{E}),
\]
where nodes are reachable states and edges represent rule applications.

\begin{figure}[H]
    \centering
    \includegraphics[width=0.95\linewidth]{multiway_futures_comparison.png}
    \caption{Comparison of multiway futures for a \textbf{glass} system (entropy only) and a \textbf{plant} system (entropy + repair). The plant produces a larger, more resilient state space.}
    \label{fig:comparison}
\end{figure}

Key topological features include:
\begin{itemize}
    \item \textbf{collapse funnels} (fragility),
    \item \textbf{branching diversity} (resilience),
    \item \textbf{recombination cycles} (adaptation).
\end{itemize}

% =========================================================
% SECTION 6 — EXAMPLE: GLASS VS PLANT
% =========================================================
\section{Example: Glass vs. Plant Systems}

We consider two contrasting rule sets:
\begin{enumerate}
    \item \textbf{Glass}: entropy only (edge deletion).  
    \item \textbf{Plant}: entropy + local repair (triangle closure).
\end{enumerate}

Both begin from a 3-node ring.
After 10 steps:
\begin{itemize}
    \item the glass system collapses to the empty graph,
    \item the plant system generates a much larger multiway graph,
    \item the plant retains nonzero connected configurations,
    \item branching and recombination are significantly higher.
\end{itemize}

These differences arise solely from rule diversity.

% =========================================================
% SECTION 7 — DISCUSSION
% =========================================================
\section{Discussion}

This framework shows that:
\begin{itemize}
    \item Resilience is not an intrinsic property; it is generated by rule diversity.
    \item Fragile systems correspond to rule sets with $\downarrow U$ and $\uparrow C$ tendencies.
    \item Resilient systems include rules that increase utility and relax constraints.
    \item The action functional becomes an emergent, macroscopic summary of rule sequences.
\end{itemize}

Multiway topology, not energy minimisation, determines the possible futures.

% =========================================================
% SECTION 8 — CONCLUSION
% =========================================================
\section{Conclusion}

We presented a discrete framework for modelling system fragility and resilience using unweighted graph dynamics.
Rules characterised by changes in Utility, Effort, and Constraints generate all futures in a multiway causal graph.
Resilience emerges from rule diversity, expressed through increased utility, relaxed constraints, and balanced effort.
This provides a generalisable, interpretable foundation for studying risk in complex systems.

\bibliographystyle{plainnat}
\bibliography{references}

\end{document}
