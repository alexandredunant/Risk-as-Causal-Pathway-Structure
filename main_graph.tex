\documentclass[11pt]{article}
\usepackage{graphicx}
\usepackage{amsmath}
\usepackage{amssymb}
\usepackage{amsthm}
\usepackage{hyperref}
\usepackage{geometry}
\usepackage{booktabs}
\usepackage{array}
\usepackage{xcolor}
\usepackage{subcaption}
\geometry{margin=1in}

\theoremstyle{definition}
\newtheorem{definition}{Definition}
\newtheorem{proposition}{Proposition}

\title{Risk as Causal Pathway Structure:\\A Rule-Based Multiway Framework}

\author{Author Names\\Institution}
\date{}

\begin{document}
\maketitle

\begin{abstract}
We propose a foundational reconceptualization of risk science based on rule-driven multiway evolution rather than action functional optimization. Instead of treating risk as the product of static factors or as trajectories selected by minimizing an action functional, we view risk as emerging from the topology of branching causal pathways generated by microscopic transformation rules. System states are represented as complete configurations (graphs), and evolution proceeds through the application of discrete rules that encode changes in Utility, Effort, and Constraints. The action functional emerges as a macroscopic summary statistic of rule applications rather than a primary selector. This perspective reveals that system behavior arises from the structure of rule sets and the resulting multiway graph topology. We distinguish between redundancy, where multiple pathways share common failure modes, and degeneracy, where pathways fail independently. By representing possible futures as multiway causal graphs where nodes are complete system configurations, we make path dependence, structural coupling, and topological transitions visible and quantifiable. This framework provides a new epistemic foundation for understanding adaptive capacity, detecting approaching tipping points, and characterizing the geometry of accessible futures.
\end{abstract}

\section{Introduction: From Classical Risk to Rule-Based Evolution}

Traditional risk analysis has long relied on a multiplicative framework:
\begin{equation}
\text{Risk} = \text{Hazard} \times \text{Exposure} \times \text{Vulnerability}.
\end{equation}

While this triplet formulation has proven useful for static risk assessment, it faces fundamental limitations when analyzing systems that evolve under deep uncertainty. Real systems are dynamic processes that evolve through time along branching pathways shaped by local transformation rules rather than global optimization principles.

\subsection{The Epistemic Shift}

We propose a fundamental reconceptualization based on three key insights:

\textbf{1. States are Complete Configurations.} System states are not simple points in abstract state space but complete configurations with rich internal structure. In graph-based systems, a state is an entire network topology $(V, E)$. This structure matters because it determines failure modes, adaptation options, and response mechanisms.

\textbf{2. Evolution is Rule-Driven, Not Optimized.} Systems do not evolve by optimizing a global action functional. Instead, they evolve through the application of local, microscopic transformation rules. These rules operate on graph topology, adding or removing nodes and edges according to local patterns and constraints.

\textbf{3. Action is Emergent, Not Prescriptive.} The action functional does not select trajectories. Rather, it emerges as a macroscopic summary of the cumulative effects of rule applications. The "cost" of a trajectory is simply the sum of changes in Utility, Effort, and Constraints along the path, arising naturally from which rules were applied.

This shifts the question from "what trajectory minimizes action?" to "what is the topology of futures accessible through rule application, and how does this topology reveal risk?"

\section{States as Graph Configurations}

\subsection{The Configuration Space}

A system state is defined as an unweighted, undirected graph:
\begin{equation}
\Sigma = (V, E)
\end{equation}
where $V$ is a fixed set of nodes and $E \subseteq V \times V$ is the set of edges. All information is encoded in the presence or absence of edges, making the representation purely topological.

\subsection{Social-Ecological-Technological Systems (SETS)}

For complex anthropogenic systems, nodes and edges represent coupled components across three domains:

\begin{itemize}
    \item \textbf{Social (S)}: Institutions, governance structures, human networks, decision-making capacity
    \item \textbf{Ecological (E)}: Ecosystems, species interactions, biogeochemical cycles, natural resources
    \item \textbf{Technological (T)}: Infrastructure, built systems, engineered networks, material flows
\end{itemize}

A configuration $\Sigma$ is not just an infrastructure state or ecological state in isolation—it is the complete tri-variate coupling between S, E, and T domains at one moment. Two systems might share identical aggregate metrics yet differ fundamentally in their structural coupling, leading to radically different fragility profiles.

\section{Rule-Based Dynamics: The Generative Grammar of Risk}

\subsection{Rules as Microscopic Transformations}

Evolution proceeds through the application of discrete transformation rules. Each rule is characterized by its effects on three fundamental dimensions:

\begin{equation}
\text{Rule}: \Sigma \rightarrow \{\Sigma_1, \Sigma_2, \ldots\} \quad \text{with} \quad (\Delta U, \Delta E, \Delta C)
\end{equation}

where:
\begin{itemize}
    \item $\Delta U$ = change in \textbf{Utility} (functional value, performance)
    \item $\Delta E$ = change in \textbf{Effort} (cost, energy, resources required)
    \item $\Delta C$ = change in \textbf{Constraints} (freedom, flexibility, operating space)
\end{itemize}

Each symbol takes values in $\{\uparrow, \downarrow, \rightarrow\}$ representing increase, decrease, or no change.

\subsection{Fundamental Rule Types}

We identify seven fundamental rule types that appear across different system classes. These rules are not domain-specific but capture universal transformation patterns in graph-based systems. Each rule is visualized as a before/after graph transformation.

\begin{figure}[htbp]
\centering
\begin{subfigure}{0.48\textwidth}
    \includegraphics[width=\textwidth]{rule_entropy.pdf}
    \caption{Entropy/Stress: Edge removal ($\Delta U \downarrow$, $\Delta E \downarrow$, $\Delta C \uparrow$). Degradation dynamics—loss of connectivity reduces capability and maintenance burden but constrains future options.}
\end{subfigure}
\hfill
\begin{subfigure}{0.48\textwidth}
    \includegraphics[width=\textwidth]{rule_repair.pdf}
    \caption{Local Repair: Triangle closure ($\Delta U \uparrow$, $\Delta E \uparrow$, $\Delta C \downarrow$). If edges $(a,b)$ and $(b,c)$ exist, add $(a,c)$. Creates redundant pathways at cost of increased complexity.}
\end{subfigure}

\vspace{0.3cm}

\begin{subfigure}{0.48\textwidth}
    \includegraphics[width=\textwidth]{rule_expansion.pdf}
    \caption{Node Expansion: Growth ($\Delta U \uparrow$, $\Delta E \uparrow$, $\Delta C \downarrow$). Add new node with edges. Increases capacity and flexibility but requires investment.}
\end{subfigure}
\hfill
\begin{subfigure}{0.48\textwidth}
    \includegraphics[width=\textwidth]{rule_pruning.pdf}
    \caption{Node Pruning: Simplification ($\Delta U \downarrow$, $\Delta E \downarrow$, $\Delta C \uparrow$). Remove node and all incident edges. Reduces both cost and capability.}
\end{subfigure}

\vspace{0.3cm}

\begin{subfigure}{0.48\textwidth}
    \includegraphics[width=\textwidth]{rule_rewiring.pdf}
    \caption{Neutral Rewiring: Reorganization ($\Delta U \rightarrow$, $\Delta E \rightarrow$, $\Delta C \rightarrow$). Replace edge $(a,b)$ with $(a,c)$. Structural change without functional change.}
\end{subfigure}
\hfill
\begin{subfigure}{0.48\textwidth}
    \includegraphics[width=\textwidth]{rule_reinforcement.pdf}
    \caption{Pattern Reinforcement: Motif completion ($\Delta U \uparrow$, $\Delta E \uparrow$, $\Delta C \downarrow$). Complete structural patterns like cliques. Strengthens robustness at increased maintenance cost.}
\end{subfigure}

\vspace{0.3cm}

\begin{subfigure}{0.48\textwidth}
    \includegraphics[width=\textwidth]{rule_suppression.pdf}
    \caption{Constraint Suppression: Degree reduction ($\Delta U \downarrow$, $\Delta E \downarrow$, $\Delta C \downarrow$). Remove multiple edges from a node. Unusual signature—all three dimensions decrease.}
\end{subfigure}

\caption{Fundamental transformation rules. Each shows before (left) and after (right) states. Green edges are added, red dashed edges are removed, green nodes are newly created. The U-E-C signature characterizes each rule's effect on the action functional.}
\label{fig:fundamental_rules}
\end{figure}

\subsection{Domain-Specific Rule Instantiations}

Table \ref{tab:domain_rules} shows how these fundamental patterns instantiate across different domains, with specific interpretations of the U-E-C changes.

\begin{table}[htbp]
\centering
\small
\begin{tabular}{p{2.5cm}p{4cm}p{1cm}p{1cm}p{1.2cm}p{4cm}}
\toprule
\textbf{Domain} & \textbf{Rule} & \textbf{$\Delta U$} & \textbf{$\Delta E$} & \textbf{$\Delta C$} & \textbf{Specific Interpretation} \\
\midrule
\multicolumn{6}{l}{\textit{Infrastructure Networks}} \\
\midrule
& Pipe rupture & $\downarrow$ & $\downarrow$ & $\uparrow$ & Loss of throughput; reduced maintenance; less flexibility \\
& Add parallel capacity & $\uparrow$ & $\uparrow$ & $\downarrow$ & Increased flow; higher capital cost; more routing options \\
& Strategic hardening & $\rightarrow$ & $\uparrow$ & $\downarrow$ & Same function; higher maintenance; more robust to shocks \\
\midrule
\multicolumn{6}{l}{\textit{Ecosystem Dynamics}} \\
\midrule
& Species extinction & $\downarrow$ & $\downarrow$ & $\uparrow$ & Loss of function; less metabolic load; reduced niche space \\
& Mutualism formation & $\uparrow$ & $\uparrow$ & $\downarrow$ & Enhanced productivity; energy investment; expanded niche \\
& Invasive arrival & $\downarrow$ & $\downarrow$ & $\uparrow$ & Disrupts natives; outcompetes locally; constrains successional paths \\
\midrule
\multicolumn{6}{l}{\textit{Social Networks}} \\
\midrule
& Tie dissolution & $\downarrow$ & $\downarrow$ & $\uparrow$ & Lost information flow; less maintenance; reduced coordination \\
& Bridge formation & $\uparrow$ & $\uparrow$ & $\downarrow$ & Connects communities; requires trust building; enables new collaborations \\
& Clustering densification & $\uparrow$ & $\uparrow$ & $\rightarrow$ & Stronger in-group bonds; more interaction effort; same external constraints \\
\midrule
\multicolumn{6}{l}{\textit{Supply Chains}} \\
\midrule
& Supplier failure & $\downarrow$ & $\downarrow$ & $\uparrow$ & Capacity loss; no procurement cost; fewer sourcing options \\
& Dual sourcing & $\uparrow$ & $\uparrow$ & $\downarrow$ & Redundancy; contract management overhead; flexibility against disruption \\
& Just-in-time consolidation & $\rightarrow$ & $\downarrow$ & $\uparrow$ & Same output; lower inventory cost; vulnerable to delays \\
\bottomrule
\end{tabular}
\caption{Domain-specific instantiations of fundamental rule patterns. Each shows how U-E-C changes manifest in particular contexts.}
\label{tab:domain_rules}
\end{table}

\section{The Action Functional as Emergent Observable}

\subsection{From Selection to Summary}

In traditional path integral formulations, the action functional determines which trajectories are probable:
\begin{equation}
P(\gamma) \propto e^{-\beta J[\gamma]}
\end{equation}

In our framework, we invert this relationship. The action does not select trajectories—rules generate all possible futures democratically. Instead, action emerges as a summary statistic:

\begin{definition}[Emergent Action]
For a trajectory $\gamma = (e_1, e_2, \ldots, e_n)$ where each $e_i$ is a rule application, the action is:
\begin{equation}
J[\gamma] = \sum_{i=1}^{n} j(e_i)
\end{equation}
where $j(e_i)$ quantifies the U-E-C changes: $j(e) = -\Delta U + \Delta E + \Delta C$.
\end{definition}

The action along a path is simply the cumulative score of rule applications. It arises naturally from which rules were applied, not from optimization. This formulation captures the intuition that:
\begin{itemize}
    \item Trajectories with many utility-increasing rules have lower action (more favorable)
    \item Trajectories with high effort or constraint accumulation have higher action (less favorable)
    \item The relative importance of U, E, and C can be tuned through coefficients if needed
\end{itemize}

\subsection{Connecting to Optimization Principles}

While systems do not optimize action globally, optimization pressures may influence which rules are applied more frequently in practice. For instance:
\begin{itemize}
    \item Ecosystems may favor energy-efficient pathways (low $\Delta E$ rules)
    \item Engineered systems may be designed to avoid constraint-increasing rules
    \item Economic systems may preferentially apply high-utility rules
\end{itemize}

These tendencies can be incorporated through rule application probabilities, but the multiway graph itself captures all possibilities without prejudice.

\section{Multiway Causal Graphs: Mapping the Space of Futures}

\subsection{Construction Algorithm}

Starting from an initial configuration $\Sigma_0$, we generate the multiway graph $\mathcal{G}_{MW} = (\mathcal{M}, \mathcal{E})$:

\begin{enumerate}
    \item Initialize with $\mathcal{M} = \{\Sigma_0\}$, $\mathcal{E} = \emptyset$, frontier $= \{\Sigma_0\}$
    \item For each timestep $t = 1, 2, \ldots, T$:
    \begin{enumerate}
        \item For each state $\Sigma$ in frontier:
        \item For each applicable rule $r$:
        \item Generate future states $\{r(\Sigma)\}$
        \item Add new states to $\mathcal{M}$ and edges to $\mathcal{E}$
        \item Update frontier with newly discovered states
    \end{enumerate}
    \item Return $\mathcal{G}_{MW}$
\end{enumerate}

This algorithm is democratic—it applies all rules in parallel and retains all possible futures. The resulting graph is a directed acyclic graph (DAG) where:
\begin{itemize}
    \item Nodes are complete system configurations (graphs)
    \item Edges represent rule applications (labeled with the rule and its U-E-C signature)
    \item Paths through the graph are complete trajectories
    \item The graph structure reveals accessible futures and their relationships
\end{itemize}

\subsection{The Geometry of Risk}

Risk is not a property of individual nodes but of the graph's global topology:

\begin{itemize}
    \item \textbf{Funnels}: Regions where many paths converge to few states, indicating vulnerability to constraint accumulation
    \item \textbf{Branching diversity}: Regions with high out-degree, indicating adaptive capacity
    \item \textbf{Recombination cycles}: States reachable through multiple independent paths, indicating genuine degeneracy
    \item \textbf{Isolated components}: Disconnected subgraphs representing alternative stable states with no transitions between them
    \item \textbf{Bottlenecks}: Critical configurations that many desirable futures must pass through
\end{itemize}

These topological features are invisible to classical risk assessment but become immediately apparent in the multiway representation.

\section{Example: Contrasting Rule Set Compositions}

\subsection{Two Systems, Different Grammars}

To illustrate how rule set composition determines multiway topology, we examine two systems evolving from identical initial conditions—a ring graph of three nodes with edges $\{(1,2), (2,3), (3,1)\}$. The systems differ only in their transformation rules:

\textbf{System A (Degradation Only):}
\begin{itemize}
    \item Rule: Entropy/Stress only
    \item Effect: Can only delete edges, one at a time
    \item U-E-C signature: ($\downarrow$, $\downarrow$, $\uparrow$)
\end{itemize}

\textbf{System B (Degradation + Repair):}
\begin{itemize}
    \item Rule 1: Entropy/Stress (as above)
    \item Rule 2: Local Repair—if edges $(a,b)$ and $(b,c)$ exist, can add $(a,c)$
    \item Effect: Can both lose and gain edges
    \item Repair signature: ($\uparrow$, $\uparrow$, $\downarrow$)
\end{itemize}

\subsection{Emergent Multiway Topologies}

After 10 timesteps, the two systems produce dramatically different multiway graphs:

\textbf{System A:}
\begin{itemize}
    \item Multiway graph: 4 states, 3 transitions
    \item Terminal state: Empty graph (complete connectivity loss) reached by step 3
    \item All paths funnel to the empty state
    \item Survival rate: 0\%
    \item Mean connectivity: 1.0 edges (monotonic decay)
    \item Action trend: Monotonically increasing along all paths
\end{itemize}

\textbf{System B:}
\begin{itemize}
    \item Multiway graph: 89 states, 256 transitions
    \item Final states: Multiple configurations with 3-6 edges persist
    \item Rich branching and recombination cycles
    \item Survival rate: 87\%
    \item Mean connectivity: 3.2 edges (maintained through repair)
    \item Action trend: Fluctuates but remains bounded
\end{itemize}

\subsection{Rule Diversity Shapes Topology}

The dramatic difference arises solely from rule set composition. System A has only degradation dynamics—once edges are lost, they can never return. The multiway graph inevitably funnels toward complete collapse, forming a directed tree that terminates in the empty state.

System B has both degradation and repair mechanisms. Lost edges can be replaced through local reorganization exploiting existing structure. The multiway graph maintains rich connectivity with multiple pathways to configurations that retain function. Crucially, many states can be reached through multiple independent paths, indicating genuine degeneracy rather than redundant copies of the same failure mode.

This demonstrates that system behavior emerges from the generative grammar encoded in the rule set. Systems with richer rule sets that balance utility-decreasing and utility-increasing transformations naturally generate multiway topologies with more extensive state space exploration and multiple viable trajectories.

\subsection{Beyond Binary Classifications}

It is important to note that these are not "good" and "bad" systems, nor are they fundamentally "fragile" and "resilient" in some essential way. They are simply two different rule sets that happen to generate very different multiway topologies from the same initial condition.

Real systems exhibit rule sets of varying complexity:
\begin{itemize}
    \item Purely degrading systems (like System A) might model unmanaged infrastructure decay, species in unsuitable habitats, or institutions without renewal mechanisms
    \item Mixed systems (like System B) might model ecosystems with natural succession, maintained infrastructure, or adaptive institutions
    \item More complex systems might have conditional rules, threshold-triggered dynamics, stochastic elements, or cross-domain coupling
\end{itemize}

The framework's power lies not in classifying systems into categories but in characterizing the multiway topology that emerges from any given rule set. Two systems with similar aggregate statistics (mean connectivity, total edge count) might have completely different multiway structures depending on their rules, leading to qualitatively different pathway distributions and risk profiles.

\section{Measuring Fragility Through Graph Topology}

\subsection{Effective Number of Paths}

Rather than counting paths, we quantify diversity through entropy:
\begin{equation}
H = -\sum_{\gamma} P(\gamma) \ln P(\gamma)
\end{equation}
yielding an effective path count:
\begin{equation}
N_{\text{eff}} = e^H
\end{equation}

This represents how many equally probable paths would generate the observed entropy. A system with $N_{\text{eff}} = 1$ is deterministic. A system with $N_{\text{eff}} = 100$ behaves as if it has one hundred equiprobable futures.

\subsection{Redundancy versus Degeneracy}

Effective path count alone provides an incomplete picture. We must distinguish:

\textbf{Redundancy}: Multiple pathways that share common failure modes. In the multiway graph, redundant paths remain close in configuration space, passing through nodes with similar structural properties. They fail together under stress because they rely on the same mechanisms.

\textbf{Degeneracy}: Multiple structurally different pathways that achieve equivalent function through independent mechanisms. In the multiway graph, degenerate pathways diverge widely, passing through configurations with different topologies. They fail independently because they operate through different mechanisms.

Visual inspection of the multiway graph immediately reveals whether apparent pathway diversity reflects genuine independence (degeneracy) or illusory correlation (redundancy).

\subsection{Topological Sensitivity}

A system might currently possess many viable pathways yet be structurally unstable, poised to undergo rapid reorganization. Fisher information quantifies this sensitivity:
\begin{equation}
I(\theta) = \mathbb{E}\left[\left(\frac{\partial \ln P(\gamma|\theta)}{\partial \theta}\right)^2\right]
\end{equation}

High Fisher information indicates that small changes in parameters cause large reorganization of which pathways remain viable. The multiway graph undergoes topological phase transitions where connectivity patterns, bottleneck locations, and pathway diversity transform dramatically.

This provides an early warning signal complementary to pathway counting. A system might have high $N_{\text{eff}}$ today but also high Fisher information, signaling that this diversity is structurally fragile.

\subsection{Spectral Properties}

The second eigenvalue of the graph Laplacian measures overall connectivity. As this spectral gap shrinks toward zero, the graph approaches fragmentation. Monitoring spectral properties provides real-time indicators of approaching critical transitions.

\section{Extending the Framework: Advanced Rule Types}

Table \ref{tab:advanced_rules} presents more sophisticated rule types that enable richer dynamics and better capture complex system behaviors.

\begin{table}[htbp]
\centering
\small
\begin{tabular}{p{3cm}p{4cm}p{1cm}p{1cm}p{1.2cm}p{4cm}}
\toprule
\textbf{Rule Category} & \textbf{Operation} & \textbf{$\Delta U$} & \textbf{$\Delta E$} & \textbf{$\Delta C$} & \textbf{Application Context} \\
\midrule
\multicolumn{6}{l}{\textit{Conditional Rules (State-Dependent)}} \\
\midrule
Threshold-triggered growth & If $|E| < k$, add node & $\uparrow$ & $\uparrow$ & $\downarrow$ & Growth only below capacity limit \\
Overload-triggered failure & If degree$(v) > d_{max}$, remove edges from $v$ & $\downarrow$ & $\downarrow$ & $\uparrow$ & Cascading failure from congestion \\
Diversity-dependent repair & If $|V|/|E| > r$, close triangle & $\uparrow$ & $\uparrow$ & $\downarrow$ & Repair only in sparse regions \\
\midrule
\multicolumn{6}{l}{\textit{Non-Local Rules (Global Coordination)}} \\
\midrule
Spanning tree pruning & Remove edge not in MST & $\rightarrow$ & $\downarrow$ & $\uparrow$ & Efficiency optimization \\
Hub reinforcement & Add edges to highest-degree nodes & $\uparrow$ & $\uparrow$ & $\downarrow$ & Preferential attachment \\
Periphery pruning & Remove low-degree nodes & $\downarrow$ & $\downarrow$ & $\rightarrow$ & Core-periphery emergence \\
\midrule
\multicolumn{6}{l}{\textit{Multi-Step Rules (Temporal Composition)}} \\
\midrule
Replace-then-repair & Remove $(a,b)$, then add $(a,c)$ and $(b,c)$ & $\uparrow$ & $\uparrow$ & $\downarrow$ & Adaptive rewiring \\
Prune-then-consolidate & Remove leaf nodes, then close triangles & $\rightarrow$ & $\downarrow$ & $\uparrow$ & Efficiency through consolidation \\
\midrule
\multicolumn{6}{l}{\textit{Stochastic Rules (Probabilistic Application)}} \\
\midrule
Random edge deletion & Delete random edge with probability $p$ & $\downarrow$ & $\downarrow$ & $\uparrow$ & Environmental noise \\
Probabilistic repair & With probability $q$, apply local repair & $\uparrow$ & $\uparrow$ & $\downarrow$ & Stochastic adaptation \\
\midrule
\multicolumn{6}{l}{\textit{Coupled Rules (Cross-Domain)}} \\
\midrule
Social-ecological feedback & If $E$-node fails, remove adjacent $S$-edges & $\downarrow$ & $\downarrow$ & $\uparrow$ & Ecosystem service loss impacts society \\
Techno-social investment & If $S$-connectivity high, add $T$-capacity & $\uparrow$ & $\uparrow$ & $\downarrow$ & Social capital enables infrastructure \\
\bottomrule
\end{tabular}
\caption{Advanced rule types extending the basic framework. These enable richer dynamics including conditional application, non-local coordination, temporal composition, stochasticity, and cross-domain coupling.}
\label{tab:advanced_rules}
\end{table}

\section{Implications Across Domains}

\subsection{Climate Adaptation}

Climate configurations include temperature fields, ocean states, circulation patterns, and ENSO phases. Rules represent natural dynamics (AMOC weakening, ice sheet loss) and human interventions (emissions reductions, adaptation investments).

The multiway graph reveals:
\begin{itemize}
    \item Which pathways maintain strong circulation vs. exhibit AMOC collapse
    \item Where decision points enable divergence to radically different climate futures
    \item Whether apparent adaptation options are genuinely independent or share vulnerabilities
    \item Early warnings of approaching tipping points through topological sensitivity
\end{itemize}

\subsection{Infrastructure Resilience}

Infrastructure configurations are complete network states including topology, capacities, flows, and component ages. Rules represent degradation, upgrades, and disruptions.

The multiway graph reveals:
\begin{itemize}
    \item Whether redundant routes provide genuine degeneracy or share failure modes
    \item Critical bottleneck configurations that many service-level futures must pass through
    \item How investment decisions constrain or expand future adaptation options
    \item Approaching percolation thresholds through spectral monitoring
\end{itemize}

\subsection{Ecosystem Dynamics}

Ecosystem configurations are spatial vegetation patterns including species composition, biomass distribution, and connectivity. Rules represent disturbances, succession, invasions, and management interventions.

The multiway graph reveals:
\begin{itemize}
    \item Alternative stable states as disconnected components
    \item Whether multiple successional pathways genuinely lead to equivalent function
    \item Critical early-stage decisions that irreversibly commit to particular outcomes
    \item How restoration interventions steer trajectories through configuration space
\end{itemize}

\section{Toward Practical Implementation}

\subsection{Computational Strategies}

\textbf{State Hashing}: Use frozensets or canonical graph representations to efficiently detect when the same configuration is reached through different paths.

\textbf{Pruning}: Apply Granger causality to identify edges that carry genuine predictive information vs. spurious correlations.

\textbf{Coarse-Graining}: Group configurations into functional equivalence classes to manage state space explosion while preserving essential topology.

\textbf{Parallelization}: Exploit the embarrassingly parallel nature of rule application across frontier states.

\subsection{Visualization}

\textbf{Layer-Based Layout}: Use time layers to show temporal evolution, with nodes colored by survival status or connectivity.

\textbf{Animation}: Create temporal evolution GIFs showing how the multiway graph unfolds, making dynamics intuitive.

\textbf{Comparative Views}: Side-by-side visualization of different rule sets (like Glass vs. Plant) immediately conveys resilience differences.

\textbf{Metric Dashboards}: Track survival rates, connectivity evolution, and state space growth in real-time.

\section{Discussion: A New Epistemic Foundation}

This framework establishes risk as emerging from rule-driven multiway evolution rather than action functional optimization. The key epistemic shifts include:

\textbf{From Optimization to Generation}: Systems do not optimize global functionals but apply local rules. The multiway graph captures all possible futures democratically.

\textbf{From Selection to Summary}: Action does not select trajectories but emerges as a macroscopic summary of cumulative rule effects.

\textbf{From Properties to Topology}: Risk is not a property of states but of the multiway graph's global geometry—its funnels, branches, bottlenecks, and connectedness.

\textbf{From Static to Dynamic}: Classical risk treats systems as static objects. We recognize them as evolving processes navigating configuration space through rule application.

\textbf{From Redundancy to Degeneracy}: Counting options is insufficient. We must distinguish genuine independence (degenerate pathways) from illusory diversity (redundant pathways).

\subsection{Characteristic Topological Patterns}

Across domains, analysis of multiway graphs reveals characteristic topological patterns associated with different rule set compositions:

\textbf{Degradation-Dominated Systems} exhibit:
\begin{itemize}
    \item Rule sets dominated by utility-decreasing transformations
    \item Multiway graphs that are sparse with strong funneling toward low-connectivity states
    \item Pathways that share failure modes (redundancy rather than degeneracy)
    \item High topological sensitivity (high Fisher information)
    \item Tight structure-function coupling (low emergence)
\end{itemize}

\textbf{Balanced Systems} exhibit:
\begin{itemize}
    \item Rule sets with both degradation and repair mechanisms
    \item Multiway graphs that are richly connected with diverse branching
    \item Pathways that diverge in configuration space (genuine degeneracy)
    \item Low topological sensitivity (stable connectivity patterns)
    \item Loose structure-function coupling (high emergence—multiple structures achieve equivalent function)
\end{itemize}

\textbf{Adaptive Systems} gain options under stress:
\begin{itemize}
    \item Rules triggered by perturbations that increase branching rather than pruning
    \item Multiway graphs that become richer under disturbance
    \item Stress opens previously inaccessible transitions, connecting formerly isolated components
    \item Create new pathways through configuration space rather than eliminating existing ones
\end{itemize}

These patterns are not value judgments—a degradation-dominated system is not inherently undesirable. In some contexts, rapid simplification might be appropriate (planned degrowth, intentional disinvestment, strategic withdrawal). The framework characterizes what multiway topology emerges from any given rule set, allowing analysts to understand the geometry of accessible futures without imposing normative classifications.

\section{Conclusion}

We have presented a rule-based multiway framework for risk science that inverts the traditional relationship between action functionals and system evolution. Rather than treating action as a selector that determines which trajectories systems follow, we recognize that:

\begin{enumerate}
    \item Systems evolve through local transformation rules
    \item Rules generate all possible futures in a multiway graph
    \item Action emerges as a summary statistic of rule applications
    \item Risk is the topology of this multiway graph
\end{enumerate}

This perspective provides both theoretical clarity and computational tractability. The multiway graph representation makes abstract pathway distributions concrete and visualizable. Stakeholders can see branching futures, identify critical decision points, and distinguish genuine from illusory robustness.

The framework does not predict which specific trajectory will be realized—this is impossible in non-ergodic systems. Instead, it characterizes the topology of accessible futures: how many pathways exist, how independent they are, how sensitive the structure is to perturbations, and where critical transitions lie.

This represents a fundamental shift from probabilistic point prediction toward topological possibility characterization. Risk science has long sought to move beyond the limitations of hazard-exposure-vulnerability multiplication while retaining rigor. The rule-based multiway framework offers such an alternative by grounding risk in the geometry of configuration spaces explored through microscopic transformation rules.

Future work will extend this foundation through topology-preserving coarse-graining methods, connections to computational irreducibility and algorithmic information, efficient algorithms for large-scale multiway graphs, machine learning for discovering optimal rule representations, validation against historical case studies, and development of decision support tools for navigating multiway spaces.

The essential insight—that risk arises from rule-driven multiway topology rather than action functional optimization—provides an organizing principle for understanding how complex systems navigate branching futures under deep uncertainty.

\bibliographystyle{plain}
\begin{thebibliography}{99}

\bibitem{kaplan1981quantitative}
Stanley Kaplan and B. John Garrick.
On the quantitative definition of risk.
\textit{Risk Analysis}, 1(1):11--27, 1981.

\bibitem{shannon1948mathematical}
Claude E. Shannon.
A mathematical theory of communication.
\textit{The Bell System Technical Journal}, 27(3):379--423, 1948.

\bibitem{holling1973resilience}
Crawford S. Holling.
Resilience and stability of ecological systems.
\textit{Annual Review of Ecology and Systematics}, 4(1):1--23, 1973.

\bibitem{mcphearson2016scientists}
Timon McPhearson, Steward T. A. Pickett, et al.
Scientists must have a say in the future of cities.
\textit{BioScience}, 66(3):175--182, 2016.

\bibitem{markolf2018interdependent}
Samuel A. Markolf, Mikhail V. Chester, et al.
Interdependent infrastructure as linked social, ecological, and technological systems (SETSs) to address lock-in and enhance resilience.
\textit{Earth's Future}, 6(12):1638--1659, 2018.

\bibitem{krueger2019requisite}
Elisabetta Krueger et al.
Requisite resilience: Aligning internal variety with environmental variety in complex adaptive systems.
\textit{Urban Trends Working Papers}, 2019.

\bibitem{wolfram2020class}
Stephen Wolfram.
A Class of Models with the Potential to Represent Fundamental Physics.
\textit{Wolfram Media}, 2020.

\bibitem{bak1987self}
Per Bak, Chao Tang, and Kurt Wiesenfeld.
Self-organized criticality: An explanation of 1/f noise.
\textit{Physical Review Letters}, 59(4):381, 1987.

\bibitem{friston2010free}
Karl Friston.
The free-energy principle: a unified brain theory?
\textit{Nature Reviews Neuroscience}, 11(2):127--138, 2010.

\bibitem{wolfram2002}
Stephen Wolfram.
\textit{A New Kind of Science}.
Wolfram Media, 2002.

\bibitem{tononi1999measures}
Giulio Tononi, Olaf Sporns, and Gerald M Edelman.
Measures of degeneracy and redundancy in biological networks.
\textit{Proceedings of the National Academy of Sciences}, 96(6):3257--3262, 1999.

\bibitem{edelman2001degeneracy}
Gerald M Edelman and Joseph A Gally.
Degeneracy and complexity in biological systems.
\textit{Proceedings of the National Academy of Sciences}, 98(24):13763--13768, 2001.

\bibitem{scheffer2009early}
Marten Scheffer et al.
Early-warning signals for critical transitions.
\textit{Nature}, 461(7260):53--59, 2009.

\bibitem{newman2010}
Mark Newman.
\textit{Networks: An Introduction}.
Oxford University Press, 2010.

\end{thebibliography}
\end{document}