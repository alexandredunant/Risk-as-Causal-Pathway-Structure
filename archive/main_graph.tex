\documentclass[11pt]{article}
\usepackage{graphicx}
\usepackage{amsmath}
\usepackage{amssymb}
\usepackage{amsthm}
\usepackage{hyperref}
\usepackage{geometry}
\usepackage{booktabs}
\usepackage{array}
\usepackage{xcolor}
\usepackage{subcaption}
\usepackage{longtable}
\usepackage{cite}
\usepackage{float}

\geometry{margin=1in}

\theoremstyle{definition}
\newtheorem{definition}{Definition}
\newtheorem{proposition}{Proposition}

\title{Risk as Causal Pathway Structure}

\author{Author Names\\Institution}
\date{}

\begin{document}
\maketitle

\begin{abstract}
We propose a foundational reconceptualization of risk science. Rather than treating risk as the product of static factors or as trajectories selected by global optimization, we view risk as emerging from the structure of branching causal pathways generated by microscopic transformation rules. This path-distributional perspective reveals that fragility arises not merely from the probability of adverse events, but from the topology of how possible futures relate to one another. We rigorously define the "action functional" not as a prescriptive force, but as a macroscopic summary statistic of local rule interactions. Furthermore, we introduce a suite of topological metrics—including pathway entropy, Fisher information, and the spectral gap—to quantify systemic resilience. These tools allow us to distinguish between \textit{redundancy}, where coupled pathways share common failure modes, and \textit{degeneracy}, where pathways provide genuine structural independence. By representing possible futures as multiway causal graphs, we provide a new epistemic foundation for distinguishing apparent from genuine robustness in complex systems evolving under deep uncertainty.
\end{abstract}

\section{Introduction: The Generative Grammar of Risk}

Traditional risk analysis has long relied on a multiplicative framework, formalized as the triplet $\text{Risk} = \text{Hazard} \times \text{Exposure} \times \text{Vulnerability}$ \cite{kaplan1981quantitative}. While this formulation provides a necessary heuristic for static assessment, it faces fundamental epistemic limitations when applied to complex systems evolving under deep uncertainty. Real systems—whether ecological networks, infrastructure grids, or social institutions—are dynamic processes that evolve along branching pathways shaped by local constraints and adaptive decisions.

We propose a shift from this static view to a rule-based evolutionary framework. In contrast to classical variational approaches where trajectories are assumed to be selected by optimizing a global functional (e.g., the principle of least action), we posit that complex anthropogenic systems evolve through a "generative grammar" of microscopic transformation rules \cite{wolfram2002}. In this view, risk is not a property of a specific state (node), but a topological feature of the multiway causal graph—the map of all possible futures generated by the iterative application of system rules.

This framework rests on three key insights. First, system states must be represented as complete configurations with internal topological structure \cite{markolf2018interdependent}. Second, evolution is driven by local, discrete rules that add, remove, or rewire components based on immediate context. Third, the "cost" or "action" of a trajectory is an emergent property arising from the cumulative effects of these rule applications, providing a summary of the system's effort and utility over time.

\section{Formalism: States and Rules}

\subsection{The Configuration Space}
We define a system state $\Sigma$ as a graph configuration $\Sigma = (V, E)$, where $V$ represents the set of system components and $E$ represents their active couplings. In the context of Social-Ecological-Technological Systems (SETS), these components span domains: social institutions ($S$), ecological assets ($E$), and technological infrastructure ($T$). A state is therefore not a simple scalar value but a complete specification of the tri-variate coupling between these domains.

\subsection{Microscopic Transformation Rules}
Evolution proceeds through the application of a set of transformation rules $\mathcal{R}$. Each rule $r \in \mathcal{R}$ acts as a local operator $\Sigma \to \{\Sigma'_1, \dots, \Sigma'_k\}$, modifying the graph topology. Crucially, each rule is characterized by a signature vector describing its marginal impact on the system's thermodynamic and functional state:
\begin{equation}
\vec{\delta}_r = (\Delta U, \Delta E, \Delta C)
\end{equation}
where $\Delta U$ represents the change in Utility (performance), $\Delta E$ the change in Effort (metabolic or economic cost), and $\Delta C$ the change in Constraints (flexibility).

Table \ref{tab:domain_rules} links these abstract topological operators to domain-specific phenomena.

\begin{table}[H]
\centering
\small
\renewcommand{\arraystretch}{1.2}
\caption{\textbf{Mapping of domain-specific rules to topological operators.} The U-E-C signature ($\Delta$Utility, $\Delta$Effort, $\Delta$Constraints) links physical events to graph evolution.}
\vspace{0.2cm}
\begin{tabular}{p{2.0cm}p{3.2cm}p{3.8cm}ccc}
\toprule
\textbf{Domain} & \textbf{Rule} & \textbf{Topological Operator} & \textbf{$\Delta U$} & \textbf{$\Delta E$} & \textbf{$\Delta C$} \\
\midrule
\textbf{Infra.} & Pipe rupture & \textbf{Edge Deletion} $(u,v) \to \emptyset$ & $\downarrow$ & $\downarrow$ & $\uparrow$ \\
& Parallel capacity & \textbf{Edge Creation} $\emptyset \to (u,v)$ & $\uparrow$ & $\uparrow$ & $\downarrow$ \\
& Grid Extension & \textbf{Node Creation} $\emptyset \to v$ & $\uparrow$ & $\uparrow$ & $\downarrow$ \\
\midrule
\textbf{Eco.} & Extinction & \textbf{Node Deletion} $v \to \emptyset$ & $\downarrow$ & $\downarrow$ & $\uparrow$ \\
& Mutualism & \textbf{Edge Creation} $\emptyset \to (u,v)$ & $\uparrow$ & $\uparrow$ & $\downarrow$ \\
& Invasion & \textbf{Node Creation} $\emptyset \to v$ & $\downarrow$ & $\downarrow$ & $\uparrow$ \\
\midrule
\textbf{Social} & Tie dissolution & \textbf{Edge Deletion} $(u,v) \to \emptyset$ & $\downarrow$ & $\downarrow$ & $\uparrow$ \\
& Bridging & \textbf{Edge Creation} $\emptyset \to (u,v)$ & $\uparrow$ & $\uparrow$ & $\downarrow$ \\
& Institution Bldg. & \textbf{Node Creation} $\emptyset \to v$ & $\uparrow$ & $\uparrow$ & $\rightarrow$ \\
\bottomrule
\end{tabular}
\label{tab:domain_rules}
\end{table}

\section{The Action Functional as Emergent Observable}

In classical physics, the action functional $S$ determines the path of a system via the principle of least action ($\delta S = 0$). In our risk framework, we invert this relationship. The system explores the phase space via the stochastic or decision-driven application of local rules. The "action" is the recorded history of these transactions.

\begin{definition}[Emergent Action Functional]
For a trajectory $\gamma$ consisting of a sequence of rule applications $(e_1, e_2, \dots, e_n)$, the accumulated action $J[\gamma]$ is the summation of the net "cost" of transformations along the path:
\begin{equation}
J[\gamma] = \sum_{i=1}^n \left( \Delta E(e_i) - \Delta U(e_i) + \Delta C(e_i) \right).
\end{equation}
\end{definition}

This functional serves as a macroscopic summary statistic. Trajectories characterized by degradation (Edge Deletion rules where $\Delta U \downarrow$ and $\Delta C \uparrow$) will exhibit a monotonically increasing action $J$. Trajectories characterized by adaptation (Edge Creation rules where $\Delta U \uparrow$) may lower or stabilize $J$.

\section{The Multiway Causal Graph}

Starting from an initial configuration $\Sigma_0$, the recursive application of the rule set generates a Multiway Causal Graph $\mathcal{G}_{MW}$. This directed acyclic graph maps the entire space of accessible futures. Nodes represent complete system configurations, while edges represent specific rule applications.

\subsection{Quantifying Topological Risk}

Once the multiway graph is constructed, we can evaluate risk not by summing static probabilities, but by analyzing the graph's topology. We propose three primary metrics for this purpose.

\subsection{Effective Path Diversity (Entropy)}
A robust system must possess a diversity of viable futures. We quantify this using the Shannon entropy of the pathway distribution.
\begin{equation}
H = -\sum_{\gamma} P(\gamma) \ln P(\gamma)
\end{equation}
The \textbf{Effective Number of Paths} is given by $N_{\text{eff}} = e^H$. A system with $N_{\text{eff}} \approx 1$ is locked into a single deterministic trajectory (often a collapse scenario), whereas high $N_{\text{eff}}$ indicates a high volume of accessible state space.

\subsection{Topological Sensitivity (Fisher Information)}
To detect approaching tipping points, we measure the stability of the pathway distribution against perturbations. The Fisher Information metric quantifies how rapidly the structure of valid futures changes in response to a shift in constraints $\theta$:
\begin{equation}
I(\theta) = \sum_{\gamma} P(\gamma) \left( \frac{\partial \ln P(\gamma)}{\partial \theta} \right)^2
\end{equation}
A spike in $I(\theta)$ indicates a topological phase transition: a small change in external conditions (e.g., resource availability) causes a massive reorganization of the multiway graph, such as the sudden closure of previously viable branches (a funneling event).

\subsection{Structural Connectivity (Spectral Gap)}
The global connectivity of the future state space is captured by the spectrum of the graph Laplacian operator $\mathbf{L}$. Specifically, the second smallest eigenvalue (the Fiedler value), $\lambda_2$, measures the algebraic connectivity (spectral gap).
\begin{equation}
\lambda_2 > 0 \implies \text{Connected Graph}
\end{equation}
As $\lambda_2 \to 0$, the multiway graph approaches fragmentation. In risk terms, this signals the emergence of isolated components—alternative stable states (e.g., "collapsed" vs. "thriving") between which no transition is possible. A resilient system maintains a large spectral gap, ensuring that recovery pathways remain accessible from any configuration.

\section{Case Study: Comparative Multiway Dynamics}

To illustrate the framework, we compare two archetypal regimes evolving from an identical initial ring-graph configuration. We utilize "Glass" and "Plant" as heuristic labels for degradation-dominant and adaptive regimes, respectively.

\subsection{System Definitions}
The \textbf{Glass} regime represents systems lacking internal maintenance mechanisms. It is governed by a rule set dominated by the \textit{Entropy} operator (Edge Deletion, $\Delta U \downarrow$).

The \textbf{Plant} regime represents adaptive systems capable of structural renewal. It employs a balanced rule set that opposes entropy with a \textit{Repair} operator (Edge Creation, $\Delta U \uparrow$), specifically exploiting local transitivity (triangle closure) to restore connectivity.

\subsection{Topological Divergence}
The divergence in the resulting multiway graphs illustrates two fundamentally different geometries of risk (Figure \ref{fig:comparison}). 

\begin{figure}[H]
    \centering
    \includegraphics[width=\textwidth]{multiway_futures_comparison.png}
    \caption{Comparative evolution of the Glass (left) and Plant (right) multiway causal graphs. The Glass system exhibits a collapsing topology where all paths converge to a single failure state. The Plant system exhibits a rich, recombining topology with high degeneracy, maintaining multiple viable configurations despite identical initial hazard and exposure.}
    \label{fig:comparison}
\end{figure}

\begin{figure}[H]
    \centering
    \includegraphics[width=0.8\textwidth]{topological_risk_table.png}
    \caption{Quantification of the topological analysis of the two systems "Glass" and "Plant".}
    \label{fig:metrics}
\end{figure}


The degradation-dominant regime ("Glass") exhibits a partial mesh topology. The phase space contracts as options are irreversibly pruned. Trajectories bundle together, inevitably converging on a single, disconnected attractor state. The emergent action functional $J[\gamma]$ increases monotonically, reflecting the unchecked accumulation of constraints.

Conversely, the adaptive regime ("Plant") generates a dense, highly connected graph. The interplay of deletion and creation rules produces recombination cycles—loops in the causal graph where the system can return to or approximate previous states of high utility. This topology exhibits high degeneracy: there are multiple structurally distinct pathways that maintain functional connectivity. This structural independence—functional redundancy without structural identity—is the topological signature of genuine resilience.

\subsection{Causal Invariance and the Topology of Closure}

A critical feature observed in both the Glass and Plant regimes is the phenomenon of \textit{graph closure}---the tendency of divergent pathways to reconverge. In the language of rewriting systems, this property is known as \textit{confluence} or \textit{causal invariance}. It implies that the macroscopic outcome is independent of the microscopic order in which local rules are applied. However, the nature of this closure differs fundamentally between the two regimes, providing a rigorous definition of "collapse" versus "resilience."

In the Glass regime, the system exhibits \textit{trivial closure}. The multiway graph forms a topology where all branching pathways eventually terminate at a unique sink state $\Sigma_{\text{collapse}}$. While causally invariant, this closure represents a loss of state volume; the system "solves" the problem of future uncertainty by eliminating all functional configurations (i.e. the glass will inevitably break and its fucntionality collapse).

In the Plant regime, the system exhibits \textit{nontrivial closure} (or homeostatic convergence). The presence of repair rules creates recombination cycles, allowing distinct trajectories to merge back into shared functional states. For instance, a sequence of "damage $\to$ repair" ($\Sigma \to \Sigma' \to \Sigma$) yields the same final topology as a sequence of "maintenance" ($\Sigma \to \Sigma$). This mesh-like topology allows the system to compress a potentially infinite history into a finite set of equivalence classes, effectively "forgetting" perturbations that were successfully mitigated.

\subsubsection{The Risk of Non-Closure: Path Dependence and Lock-In}

The absence of closure signals a third, distinct risk regime: \textit{radical path dependence}. In a non-closed multiway graph (a purely divergent tree topology), the commutation relation between rules fails:
\[
\mathcal{R}_a(\mathcal{R}_b(\Sigma)) \neq \mathcal{R}_b(\mathcal{R}_a(\Sigma))
\]
Here, the specific sequence of events permanently constrains the future state space. In risk terms, this topology represents \textit{lock-in} or \textit{hysteresis}, where a momentary failure to apply a mitigation rule ($e.g.$, delayed maintenance) renders future recovery pathways topologically inaccessible. Unlike the Plant regime, where energy expenditures ($\Delta E$) can force divergent paths to reconverge, a non-closed system lacks the requisite transformation rules to reverse entropy. Consequently, the \textit{effective number of paths} $N_{eff}$ grows exponentially with time, rendering the system computationally irreducible and operationally unmanageable.


\section{Conclusion}

The rule-based multiway framework establishes a new epistemic foundation for risk science. By defining risk as a property of the causal graph topology rather than a static probability, we can identify latent fragilities that classical methods miss. Specifically, the framework distinguishes between apparent robustness (high path count via redundancy) and genuine resilience (high path count via degeneracy). Future work will focus on developing efficient coarse-graining algorithms to scale this topological analysis to high-dimensional SETS configurations, enabling decision-makers to navigate the complex geometry of the future.


\end{document}