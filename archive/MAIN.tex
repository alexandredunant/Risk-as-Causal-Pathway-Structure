\documentclass[12pt]{article}
\usepackage[utf8]{inputenc}
\usepackage[english]{babel}
\usepackage{amsmath}
\usepackage{amssymb}
\usepackage{graphicx}
\usepackage{geometry}
\usepackage{hyperref}
\usepackage{setspace}
\usepackage{float}
\usepackage{natbib}
\geometry{margin=1in}
\setstretch{1.2}

\title{Risk as Causal Pathway Structure: A Multiway Topological Framework for Systemic Futures}
\author{}
\date{}

\begin{document}
\maketitle

\begin{abstract}
Risk analysis traditionally treats risk as a static quantity derived from hazard, exposure, and vulnerability. This framing fails to capture the deep uncertainty and branching futures of complex Social--Ecological--Technological Systems (SETS). We propose a generative, rule-based framework in which risk emerges not from static probabilities but from the topology of the multiway causal graph---the ensemble of all possible futures generated by local transformation rules. System evolution is described as a causal grammar, with each rule modifying utility, effort, and constraints. We introduce topological metrics such as pathway entropy, Fisher information, and algebraic connectivity to quantify resilience and fragility. Through a synthetic comparison of a degradation-dominant regime (``Glass'') and an adaptive regime (``Plant''), we show how resilience corresponds to structural degeneracy and recombination cycles. This framework provides a new epistemic basis for linking causal structure to decision-relevant scenario diversity, offering principled metrics for robust decision-making under deep uncertainty.
\end{abstract}

\section{Introduction: The Generative Grammar of Risk}

The classical triplet formulation of risk (\emph{hazard} $\times$ \emph{exposure} $\times$ \emph{vulnerability}) \citep{kaplan1981quantitative, turner2003framework, birkmann2006measuring} provides useful static intuition but cannot represent the branching, path-dependent futures characteristic of complex Social--Ecological--Technological Systems (SETS). Real systems evolve through local interactions, adaptive decisions, and structural feedbacks that exhibit emergent behavior and nonlinear dynamics \citep{nicolis1977self, mitchell2009complexity}. Under conditions of deep uncertainty \citep{walker2013deep, lempert2003shaping}, risk cannot be adequately captured as a scalar probability-weighted outcome but must instead be understood as a geometrical property of the space of possible futures.

We propose a fundamental shift from static assessment to a \emph{generative} viewpoint, wherein risk emerges as a property of the \emph{multiway causal graph} generated by the iterative application of local transformation rules. This perspective emphasizes the structure, connectivity, and convergence properties of possible futures, enabling risk evaluation through topological analysis rather than static aggregate measures. The multiway approach draws inspiration from path integral formulations in quantum mechanics \citep{feynman1948spacetime, feynman1965quantum}, where all possible paths contribute to system evolution, though we do not invoke quantum principles per se. Rather, we adapt this mathematical structure to represent the ensemble of possible trajectories in complex adaptive systems.

Our approach is inspired by multiple conceptual traditions. From computer science and formal methods, we draw on rewriting systems and term algebras \citep{wolfram2002}. From network science, we leverage tools for analyzing graph topology, spectral properties, and cascade dynamics \citep{buldyrev2010catastrophic}. From ecology and resilience theory \citep{holling1973resilience, walker2004resilience, scheffer2009critical}, we adopt the understanding that system robustness emerges from functional redundancy and adaptive capacity. From climate science, we build on the storyline approach to scenario design \citep{shepherd2018storylines, sillmann2021event}, which emphasizes physically consistent narratives over probabilistic aggregation. Finally, from economics and complex systems theory, we incorporate insights about path dependence, lock-in effects \citep{arthur1989competing, david1985clio}, and the role of increasing returns in shaping long-term outcomes. Our framework aims to unify these conceptual threads into a coherent, operationalizable formalism for systemic risk analysis.

\subsection*{Epistemic Role of the Framework}

The proposed framework operates simultaneously at multiple epistemic levels, each serving distinct analytical purposes. At the most fundamental level, it provides a conceptual foundation for reformulating risk as an emergent property of causal topology rather than as a static aggregate of component vulnerabilities. This conceptual reorientation has profound implications for how we understand system fragility and resilience, shifting attention from point estimates to structural properties of the possibility space.

At the formal level, the framework constitutes a generative model based on rule-based grammars capable of expressing the evolution of SETS through local transformation operators. This formalism provides a precise mathematical language for describing how systems transition between configurations and how different causal pathways interact, converge, or diverge. The rule-based structure allows for compositional reasoning about system evolution while maintaining tractability through local specification of dynamics.

At the computational level, when transformation rules are empirically calibrated to specific systems, the framework becomes a predictive simulation tool capable of generating concrete scenario ensembles for decision support. This instantiation enables quantitative analysis of intervention strategies, sensitivity to parameter variations, and identification of critical transition points. However, we emphasize that the framework's value extends beyond prediction to include conceptual clarification and formal reasoning about risk structure.

Our goal is not to impose a single epistemic interpretation but to provide a flexible formalism that supports conceptual reasoning, algorithmic exploration, and empirical grounding in complementary ways. Different applications may emphasize different aspects of this multi-level structure depending on data availability, stakeholder needs, and the specific decision context.

\section{Formalism: States and Rules}

\subsection{The Configuration Space}

We represent a system state as a graph configuration \(\Sigma = (V, E)\), where \(V\) denotes the set of components spanning social, ecological, and technological domains, and \(E\) represents the couplings, dependencies, and interactions between these components \citep{markolf2018interdependent}. This representation recognizes that SETS are fundamentally relational, with system behavior emerging from the pattern of connections as much as from the properties of individual elements. A state is therefore a high-dimensional configuration rather than a low-dimensional scalar descriptor, capturing both compositional and structural information about the system.

The graph representation provides several advantages for risk analysis. First, it naturally accommodates heterogeneous system components across different domains without requiring commensuration into a common metric. Second, it explicitly represents dependencies and cascading pathways that are central to systemic risk \citep{buldyrev2010catastrophic}. Third, it enables application of well-developed mathematical tools from network science and spectral graph theory. Finally, it provides intuitive visualization of system structure and evolution, facilitating stakeholder communication and participatory scenario development.

\subsection{Microscopic Transformation Rules}

System evolution is driven by a set of local transformation rules \(R = \{r_1, \ldots, r_n\}\), each of which acts as a rewriting operator on graph configurations. At their most fundamental level, these rules are purely structural operations that manipulate the graph topology through node and edge creation, deletion, or modification. Formally, each rule maps a configuration to a set of possible successor configurations:
\[
    r: \Sigma \rightarrow \{\Sigma'_1, \ldots, \Sigma'_k\}.
\]
The multiplicity of successors (\(k > 1\)) reflects genuine uncertainty about which outcome will occur when the rule is applied, capturing both aleatory uncertainty (inherent stochasticity) and epistemic uncertainty (limited knowledge about system response).

Crucially, the transformation rules themselves are defined solely by their graph rewriting operations—they specify which nodes or edges are added, removed, or modified under what conditions. The rules operate at the level of structural grammar, independent of any particular interpretation or evaluation of their consequences. This separation between operational definition and consequential evaluation is central to the framework's generality and flexibility.

However, once a rule is applied to a configuration, we can observe and quantify its effects along multiple dimensions of system behavior. We characterize these consequential observables through a signature vector:
\[
\delta \vec{r} = (\Delta U, \Delta E, \Delta C),
\]
where the components represent changes in utility or performance (\(\Delta U\)), energetic, economic, or operational effort (\(\Delta E\)), and constraints on future evolution (\(\Delta C\)). These quantities are not determinants of the rule but rather emergent properties that can be measured, calculated, or estimated once the rule is applied to a specific configuration. The same graph operation may produce different observable signatures depending on the configuration to which it is applied and the context-specific interpretation of utility, effort, and constraint.

This three-dimensional characterization of observables parallels frameworks in thermodynamics \citep{nicolis1977self}, ecological energetics, and economic production theory, while extending them to incorporate explicit representation of how actions modify the space of future possibilities. The utility component \(\Delta U\) captures immediate functional consequences of a transformation, including changes in system performance, service delivery, or goal achievement. The effort component \(\Delta E\) represents resource costs, whether energetic, material, informational, or economic. These two components resemble traditional cost-benefit analysis but are embedded within a dynamic framework that tracks cumulative effects through time.

\subsubsection*{Interpretation of \(\Delta C\)}

The constraint component \(\Delta C\) measures how a rule alters the \emph{space of possible futures}, representing a crucial but often overlooked dimension of system evolution. Positive values of \(\Delta C\) indicate increased restriction of future options, corresponding to loss of adaptive capacity, irreversible commitments, or path-narrowing lock-in \citep{arthur1989competing}. Negative values indicate increased flexibility, corresponding to creation of new options, expanded adaptive capacity, or unlocking of previously inaccessible pathways.

This notion of constraint connects to several theoretical frameworks. In thermodynamics and statistical mechanics, it relates to entropy and the accessible phase space \citep{jaynes1957information, jaynes1957information2}. In information theory \citep{shannon1948mathematical, cover2006elements}, it corresponds to changes in channel capacity or coding efficiency. In ecology, it parallels the concept of niche breadth and functional diversity \citep{oliver2015biodiversity}. In cognitive science and neuroscience, it resonates with free-energy formulations of perception and action \citep{friston2010free}, where organisms minimize surprise by maintaining access to familiar states. In economics, it relates to option value and real options theory in investment analysis.

By explicitly tracking constraints, our framework captures path dependence and hysteresis effects that are central to understanding systemic risk but difficult to represent in conventional risk assessment. A system may accept short-term gains (\(\Delta U > 0\)) at the cost of reduced future flexibility (\(\Delta C > 0\)), creating vulnerability to future perturbations. Conversely, investments in redundancy and adaptive capacity may impose immediate costs (\(\Delta U < 0\), \(\Delta E > 0\)) while expanding future options (\(\Delta C < 0\)), enhancing long-term resilience.

\subsection{Domain-Specific Examples}

Table 1 illustrates how graph rewriting operations map to observable signatures across different SETS domains. The fundamental operations—node creation/deletion and edge creation/deletion—produce measurable consequences in utility, effort, and constraint that can be interpreted in domain-specific contexts. It is important to recognize that the same graph operation (e.g., edge deletion) constitutes the same structural rule but may be interpreted as different phenomena across domains: pipe rupture in infrastructure, tie dissolution in social networks, or loss of trophic connections in ecosystems.

In infrastructure systems, edge deletion through component failure (e.g., pipe rupture) reduces performance (\(\Delta U < 0\)), eliminates maintenance costs for that component (\(\Delta E < 0\)), but restricts system flexibility by removing redundant pathways (\(\Delta C > 0\)). Conversely, adding parallel capacity through infrastructure investment increases performance (\(\Delta U > 0\)), requires operational resources (\(\Delta E > 0\)), but reduces constraints by providing alternative pathways (\(\Delta C < 0\)).

\begin{table}[H]
\centering
\begin{tabular}{l l c c c}
\hline
Domain & Rule & $\Delta U$ & $\Delta E$ & $\Delta C$ \\
\hline
Infrastructure & Pipe rupture (edge deletion) & $\downarrow$ & $\downarrow$ & $\uparrow$ \\
Infrastructure & Parallel capacity (edge creation) & $\uparrow$ & $\uparrow$ & $\downarrow$ \\
Ecological & Extinction (node deletion) & $\downarrow$ & $\downarrow$ & $\uparrow$ \\
Ecological & Mutualism (edge creation) & $\uparrow$ & $\uparrow$ & $\downarrow$ \\
Social & Tie dissolution (edge deletion) & $\downarrow$ & $\downarrow$ & $\uparrow$ \\
Social & Bridging (edge creation) & $\uparrow$ & $\uparrow$ & $\downarrow$ \\
\hline
\end{tabular}
\caption{Examples of rule signatures for SETS processes across infrastructure, ecological, and social domains. Deletion rules generally reduce utility and increase constraints, while creation rules increase utility at the cost of effort while reducing constraints.}
\end{table}

In ecological systems, species extinction represents node deletion with loss of ecosystem function (\(\Delta U < 0\)), reduced metabolic demand (\(\Delta E < 0\)), but diminished functional redundancy and adaptive capacity (\(\Delta C > 0\)). The establishment of mutualistic relationships creates new edges that enhance ecosystem productivity (\(\Delta U > 0\)), requires co-evolutionary investment (\(\Delta E > 0\)), but expands ecological possibility through functional diversification (\(\Delta C < 0\)) \citep{holling1973resilience}.

In social systems, dissolution of social ties reduces collective problem-solving capacity (\(\Delta U < 0\)), lowers interaction costs (\(\Delta E < 0\)), but restricts access to diverse information and resources (\(\Delta C > 0\)). Bridge formation across disparate groups enhances coordination potential (\(\Delta U > 0\)), demands relationship maintenance effort (\(\Delta E > 0\)), but unlocks previously inaccessible social pathways (\(\Delta C < 0\)).

\section{The Action Functional as a Macroscopic Observable}

In classical physics, the principle of least action determines system trajectories through variational optimization \citep{landau1976mechanics, arnold1989mathematical, gelfand2000calculus}. In contrast to such variational systems, we do not assume that the action functional determines which trajectory the system will follow. Real complex systems are not generally optimal in any simple sense, and their evolution is shaped by historical contingency, bounded rationality, and multi-objective tradeoffs that resist reduction to a single optimization principle.

Instead, we define a simple additive functional that records the cumulative effects of rule applications along a pathway \(\gamma\) through the multiway graph:
\[
J[\gamma] = \sum_{i=1}^{n} \left( \Delta E(e_i) - \Delta U(e_i) + \Delta C(e_i) \right).
\]
Here, \(e_i\) represents the \(i\)-th transformation in the sequence defining pathway \(\gamma\), and the sum accumulates the signature components across all transformations from initial to final state.

This functional serves as a \emph{macroscopic summary} of local transformations, providing a scalar characterization of pathways that facilitates comparison and analysis. The specific linear combination \(\Delta E - \Delta U + \Delta C\) can be interpreted as a generalized cost that penalizes effort, rewards utility, and penalizes loss of future options. However, this interpretation is not privileged by the framework. Alternative functionals could weight the components differently, incorporate nonlinear combinations, or use entirely different observables without altering the underlying multiway causal structure. The choice of functional depends on analytical goals and value judgments about tradeoffs between immediate performance, resource efficiency, and long-term adaptability.

The action functional can be related to concepts from diverse fields. In thermodynamics, it resembles free energy functionals that balance energetic costs against entropic constraints \citep{jaynes1957information}. In ecology, it parallels fitness landscapes where populations evolve toward configurations that balance growth and resource expenditure \citep{scheffer2009critical}. In economics, it echoes utility maximization under budget constraints \citep{markowitz1952portfolio}. In machine learning and inference, it relates to regularized loss functions that balance model fit against complexity \citep{akaike1974new, schwarz1978estimating, gelman2013bayesian}. These connections suggest that the action functional captures universal tradeoffs between performance, cost, and constraint that arise across multiple domains.

\section{The Multiway Causal Graph}

Starting from an initial configuration \(\Sigma_0\), recursive application of the rule set \(R\) generates the multiway causal graph \(G_{\mathrm{MW}}\), a directed graph structure where nodes represent distinct system configurations and edges represent specific rule applications transforming one configuration into another. This graph encodes the entire accessible future state space, capturing not only individual trajectories but also the relationships between different pathways, including their convergence, divergence, and potential reconnection.

The multiway structure generalizes conventional scenario trees by explicitly representing causal relationships between branches. In a simple tree, once pathways diverge they never reconnect, implying that early choices irrevocably determine long-term outcomes. In the multiway graph, pathways can converge back to common configurations through different sequences of transformations, reflecting the possibility that distinct causal sequences may lead to functionally equivalent states. This convergence property, which we term \emph{causal closure}, is central to understanding system resilience.

The multiway graph also differs fundamentally from conventional state-space representations in dynamical systems theory \citep{gardiner2009stochastic}. In continuous-state systems, the state space is typically a manifold with smooth trajectories flowing through it. In contrast, the multiway graph represents discrete configurations and transformations, with topology determined by which transitions are possible rather than by metric distance in state space. This discrete, combinatorial structure is more natural for systems where transformations are qualitative rather than quantitative, involving creation or destruction of components and relationships rather than smooth parameter changes.

The concept of multiway evolution originates in the study of formal rewriting systems and has been extensively developed in the context of fundamental physics and computation \citep{wolfram2002}. We adapt this framework to SETS by interpreting nodes as socio-ecological-technological configurations and edges as interventions, adaptations, or disturbances that transform system structure. The resulting formalism provides a natural language for expressing deep uncertainty, where the future is represented not as a probability distribution over outcomes but as a network of possibilities connected by causal pathways.

\subsection{Metrics for Topological Risk}

Quantitative risk assessment requires metrics that capture relevant features of system vulnerability and resilience. We propose three complementary metrics based on information theory, statistical physics, and spectral graph theory that together characterize the topological structure of the possibility space.

\subsubsection{Effective Path Diversity (Entropy)}

The diversity of viable pathways through the multiway graph provides a direct measure of system flexibility and adaptive capacity. We quantify this diversity using Shannon entropy \citep{shannon1948mathematical}:
\[
H = -\sum_{\gamma} P(\gamma) \ln P(\gamma),
\]
where the sum runs over all pathways \(\gamma\) from the initial state to terminal states (or to a specified time horizon), and \(P(\gamma)\) represents the probability or weight assigned to pathway \(\gamma\). The entropy \(H\) quantifies the uncertainty about which pathway will be realized, with high entropy indicating many viable alternatives and low entropy indicating convergence toward a limited set of outcomes.

To facilitate interpretation, we convert entropy to an effective number of pathways \citep{jost2006entropy, hill1973diversity}:
\[
N_{\mathrm{eff}} = e^H.
\]
This quantity has an intuitive interpretation as the number of equally probable pathways that would produce the same entropy as the actual distribution. High \(N_{\mathrm{eff}}\) indicates a rich, diverse possibility space with many structurally distinct futures, suggesting resilience through multiplicity of options. Low \(N_{\mathrm{eff}}\) indicates narrowing of possibilities, suggesting vulnerability to perturbations that block the limited available pathways.

The entropy metric connects to established concepts in ecology \citep{jost2006entropy}, where diversity indices characterize ecosystem stability and response diversity \citep{oliver2015biodiversity}. It also relates to portfolio theory in finance \citep{markowitz1952portfolio, meucci2009managing}, where diversification reduces risk by avoiding concentration in correlated assets. In the exploration-exploitation literature \citep{mehlhorn2015unpacking}, entropy measures the breadth of strategic options available to adaptive agents. Our application to systemic risk extends these ideas to the topology of causal pathways rather than diversity of components or strategies.

\subsubsection{Fisher Information (Topological Sensitivity)}

While entropy measures the overall diversity of pathways, it does not capture how sensitive this structure is to changes in system parameters or forcing conditions. To quantify this sensitivity, we employ Fisher information from statistical inference theory \citep{cover2006elements, amari2016information}:
\[
I(\theta) = \sum_{\gamma} P(\gamma) 
\left( \frac{\partial \ln P(\gamma)}{\partial \theta} \right)^2,
\]
where \(\theta\) represents a parameter that affects pathway probabilities, such as the likelihood of specific rule applications, external forcing intensity, or system properties that modulate transformation rates.

Fisher information measures how much information observations of pathway outcomes provide about the parameter \(\theta\). High Fisher information indicates that small changes in \(\theta\) substantially alter the pathway distribution, signifying a system near a topological phase transition where the structure of the possibility space is highly sensitive to conditions. Spikes in \(I(\theta)\) as functions of \(\theta\) or time indicate critical transition points \citep{scheffer2009early, dakos2012methods} where the system crosses from one qualitative regime to another.

This metric has deep connections to differential geometry and information geometry \citep{amari2016information, ruppeiner1995riemannian}, where Fisher information defines a Riemannian metric on probability distributions. In thermodynamics, divergence of information geometry signals phase transitions \citep{ruppeiner1995riemannian}. In our context, Fisher information provides early warning of impending topological transitions in the multiway graph, enabling proactive intervention before the system crosses into a degraded regime with diminished adaptive capacity.

\subsubsection{Spectral Gap (Structural Connectivity)}

The algebraic connectivity of the multiway graph characterizes how well-connected different regions of the possibility space are, measuring the ease of transitioning between alternative futures. We quantify this through the spectral gap, defined as the second-smallest eigenvalue \(\lambda_2\) of the graph Laplacian matrix \(L\):
\[
\lambda_2 \text{ of the Laplacian } L.
\]
The Laplacian encodes the structure of connections in the graph, and its spectrum reveals global properties not apparent from local inspection \citep{scheffer2009critical}.

Algebraic connectivity has a direct interpretation in terms of system resilience. High \(\lambda_2\) indicates strong connectivity, meaning that many alternative pathways exist between different system states, facilitating adaptation and recovery from perturbations. Low \(\lambda_2\) indicates fragile connectivity, where the possibility space consists of poorly connected regions that are difficult to traverse. The limiting case \(\lambda_2 \to 0\) corresponds to complete disconnection, where the multiway graph fragments into isolated components representing distinct regimes between which no transitions are possible. This topological collapse signifies catastrophic loss of adaptive capacity and irreversible commitment to degraded outcomes.

Spectral methods have proven powerful in network science for identifying vulnerable structures \citep{buldyrev2010catastrophic}, in community detection for revealing modular organization, and in graph partitioning for understanding connectivity bottlenecks. In the context of early warning signals for critical transitions \citep{scheffer2009early, dakos2012methods}, spectral properties provide indicators of impending regime shifts. Our application to the multiway causal graph extends these methods to the topology of possible futures rather than the topology of current system structure.

\subsection{Decision Relevance: Linking to Storylines}

The storyline approach to scenario design \citep{shepherd2018storylines, sillmann2021event} has emerged as a powerful framework for addressing deep uncertainty in complex systems, particularly in climate science where probabilistic predictions face fundamental limitations. Rather than aggregating outcomes into probability distributions, the storyline approach constructs physically consistent narratives that explore how different combinations of factors lead to distinct outcomes. This emphasis on causal coherence over probabilistic weighting aligns naturally with our multiway causal framework.

Our topological metrics map directly onto key concerns in storyline-based analysis. The effective number of pathways \(N_{\mathrm{eff}}\) quantifies the number of structurally distinct storylines that merit consideration, providing a principled basis for scenario selection that balances comprehensiveness against tractability. Systems with high \(N_{\mathrm{eff}}\) require exploration of many distinct futures to capture the range of possibilities, while systems with low \(N_{\mathrm{eff}}\) have naturally limited scenario diversity.

The algebraic connectivity \(\lambda_2\) measures the feasibility of transitions between storylines, indicating whether they represent distinct regimes separated by high barriers or closely related possibilities with easy interconversion. High \(\lambda_2\) suggests that scenario boundaries are porous and that intermediate futures are accessible, while low \(\lambda_2\) indicates sharp distinctions between storylines with difficult or impossible transitions. This information guides both scenario design and policy analysis, revealing which storylines represent stable attractors and which are transient states.

Fisher information \(I(\theta)\) quantifies the sensitivity of storyline structure to forcing changes, identifying which parameters or interventions most substantially alter the topology of possibilities. This guides both scientific investigation, by revealing which factors deserve careful characterization, and policy design, by identifying leverage points where interventions can shift the system toward more favorable regimes.

Together, these metrics provide actionable, decision-relevant indicators of systemic flexibility that translate directly into practical scenario analysis. They operationalize abstract notions of resilience and vulnerability in terms of concrete features of the possibility space, enabling quantitative comparison of alternative policies and system designs.

\section{Case Study: Glass and Plant Regimes}

To illustrate how different rule structures generate distinct topological regimes, we compare two archetypal systems that differ fundamentally in their balance of degradation and repair mechanisms. These stylized cases illuminate core principles about how local transformation rules scale up to determine macroscopic system behavior and long-term risk profiles.

The \textbf{Glass} regime is dominated by deletion rules representing degradation, damage, or failure processes without compensating repair or regeneration mechanisms. Rules progressively remove edges and nodes from the system graph, reducing connectivity and eliminating functional components. This regime corresponds to brittle infrastructure lacking maintenance, ecosystems experiencing extinction cascades without compensating speciation, or social networks undergoing dissolution without reformation.

The \textbf{Plant} regime maintains a balance between deletion and creation rules, with local repair mechanisms that can restore lost connections through transitivity principles. When edges are deleted, nearby nodes can form compensating connections, maintaining overall connectivity through structural reorganization. This regime corresponds to maintained infrastructure with redundancy, ecosystems with functional compensation and niche filling, or social networks with bridge-building and tie reformation.

These archetypal labels capture essential differences in adaptive capacity. Glass shatters irreversibly when struck, with cracks propagating through the material without self-repair. Plants heal damage through local growth and structural reorganization, with pruned branches stimulating compensatory sprouting. The metaphor extends to understanding systemic fragility and resilience across diverse applications.

\subsection{Topological Divergence}

Despite identical initial configurations and exposure to the same disturbances, the Glass and Plant regimes evolve dramatically different multiway graph structures. The Glass system exhibits a collapsing partial mesh topology where diverse initial pathways progressively converge to a unique sink state representing complete system failure. Early in the evolution, the multiway graph shows apparent diversity as different components fail in different sequences. However, this diversity is illusory because all pathways ultimately lead to the same degenerate outcome. The effective number of pathways \(N_{\mathrm{eff}}\) declines monotonically as the system ages, and the spectral gap \(\lambda_2\) approaches zero, signifying topological collapse.

\begin{figure}[H]
    \centering
    \includegraphics[width=\textwidth]{multiway_futures_comparison.png}
    \caption{Comparative evolution of the Glass (left) and Plant (right) multiway causal graphs over equivalent time horizons. The Glass system exhibits a collapsing topology where all paths converge to a single failure state, showing high early diversity that progressively narrows to a unique terminal configuration. The Plant system exhibits a rich, recombining topology with high degeneracy, maintaining multiple viable configurations through repair cycles that prevent convergence to failure. Despite identical initial hazard and exposure conditions, the difference in transformation rules produces radically different possibility spaces, demonstrating that risk emerges from causal topology rather than from static vulnerability metrics.}
    \label{fig:comparison}
\end{figure}

In contrast, the Plant system exhibits rich recombination cycles that produce structural degeneracy, where multiple distinct transformation sequences lead to functionally equivalent configurations. The multiway graph develops a highly connected mesh structure with numerous reconvergence points, maintaining high \(N_{\mathrm{eff}}\) and \(\lambda_2\) throughout evolution. Even as individual pathways diverge due to stochastic disturbances, repair mechanisms enable the system to explore and reconnect different regions of the configuration space.

Figure 1 visualizes this fundamental topological divergence. The contrasting structures demonstrate that risk is not determined by initial conditions or disturbance intensity alone but emerges from the topology of transformation rules. Systems exposed to identical hazards can exhibit radically different risk profiles depending on their capacity for adaptive reorganization.

\begin{figure}[H]
    \centering
    \includegraphics[width=0.8\textwidth]{topological_risk_table.png}
    \caption{Quantitative topological metrics comparing the Glass and Plant regimes. The Glass system shows monotonic decline in effective pathway diversity (\(N_{\mathrm{eff}}\)), high Fisher information near collapse indicating sensitivity to perturbations, and vanishing spectral gap (\(\lambda_2 \to 0\)) signifying loss of connectivity. The Plant system maintains high diversity, moderate sensitivity, and strong connectivity throughout evolution. These metrics quantify the intuitive distinction between brittle and resilient systems in terms of precise features of the multiway causal graph.}
    \label{fig:metrics}
\end{figure}

Figure 2 quantifies these topological differences through the metrics introduced in Section 5.1. The stark contrasts in \(N_{\mathrm{eff}}\), \(I(\theta)\), and \(\lambda_2\) provide objective, quantifiable measures of resilience and fragility that go beyond qualitative intuition to enable rigorous comparison of alternative system designs.

\subsection{Causal Invariance and Closure}

A crucial topological property distinguishing the two regimes is the degree of \emph{causal invariance}---the extent to which different sequences of transformations applied to the same initial state lead to equivalent final configurations. Systems exhibiting causal invariance possess a form of structural robustness where the order of perturbations does not fundamentally determine long-term outcomes.

In the Glass regime, causal invariance is trivially satisfied through convergence: all pathways eventually reach the same absorbing failure state regardless of the sequence of edge deletions. However, this represents the worst possible form of invariance, where insensitivity to pathway details arises from collapse of all diversity rather than from adaptive compensation. This trivial closure offers no resilience benefit because the unique endpoint is catastrophic.

The Plant regime exhibits nontrivial causal invariance through functional equivalence classes. Multiple distinct transformation sequences reunite into configurations that are structurally different but functionally equivalent, providing the same performance and adaptive capacity. For example, after suffering edge deletions in different locations, the system may form compensating connections through different mechanisms, arriving at distinct graph configurations that nonetheless support equivalent function. This nontrivial closure provides genuine resilience because the reconvergent states maintain viability rather than representing shared failure.

The concept of causal invariance connects to fundamental ideas in physics, where symmetries and conservation laws arise from invariance principles \citep{landau1976mechanics, arnold1989mathematical}. In our context, repair rules act as symmetry operations that restore certain structural properties despite local perturbations, analogous to how physical symmetries preserve quantities like energy or momentum through transformations. This analogy suggests deep connections between topological resilience and algebraic structure that merit further investigation.

\subsection{Non-Closure and Lock-In}

Systems lacking adequate repair mechanisms exhibit non-closure, where transformation order fundamentally matters because different sequences lead to irreversibly distinct outcomes. Formally, the transformation rules fail to commute:
\[
R_a(R_b(\Sigma)) \neq R_b(R_a(\Sigma)),
\]
meaning that applying rule \(R_a\) followed by rule \(R_b\) produces a different configuration than applying them in reverse order. This non-commutativity induces hysteresis and path dependence, where early choices constrain later possibilities through irreversible commitment.

In the Glass regime, edge deletion is inherently non-commutative with respect to any potential repair rules (which are absent). Once an edge is deleted, any functionality depending on that connection is permanently lost, closing off entire branches of the possibility space. The system evolves as a pure divergence tree where pathways never reconnect, and the configuration space progressively fragments into isolated, unreachable regions.

This loss of connectivity manifests as lock-in effects \citep{arthur1989competing, david1985clio} familiar from economic systems with increasing returns, technological systems with compatibility requirements, and ecological systems approaching extinction thresholds. Once locked into a degraded state, the system cannot access previously available configurations even if conditions change. The constraint parameter \(\Delta C\) accumulates positive values along all pathways, progressively restricting the accessible future.

Lock-in represents a form of option foreclosure where immediate decisions irrevocably determine long-term outcomes. In the language of finance and decision theory, the system loses option value through premature commitment to particular pathways \citep{markowitz1952portfolio}. In the language of statistical mechanics, the accessible phase space contracts through entropy reduction. In the language of information theory \citep{cover2006elements}, channel capacity declines through loss of alternative coding schemes.

The Plant regime avoids lock-in through repair rules that provide partial reversibility and alternative pathways to functionally equivalent states. While specific transformation sequences remain path-dependent in their details, the functional consequences exhibit robustness to ordering variations. This flexibility emerges from structural degeneracy, where multiple configurations can satisfy the same functional requirements, combined with repair mechanisms that enable navigation between these functionally equivalent but structurally distinct states.

\section{Conclusion}

This work proposes a fundamental reconceptualization of systemic risk as a property of the topology of possible futures rather than as a static scalar derived from component vulnerabilities. The multiway causal graph framework provides a generative mathematical structure in which resilience corresponds to structural degeneracy and adaptive reconvergence, while fragility corresponds to narrowing connectivity and irreversible lock-in. This shift from static to dynamic, from scalar to topological, and from aggregate to causal enables new forms of reasoning about systemic risk under deep uncertainty.

Our framework integrates conceptual, formal, and computational roles, operating simultaneously as a philosophical stance on the nature of risk, a mathematical formalism for precise reasoning about pathway structure, and a computational approach to scenario generation and analysis. This multi-level architecture provides flexibility for different applications while maintaining formal coherence across levels.

The topological metrics introduced—effective pathway diversity, Fisher information, and algebraic connectivity—provide decision-relevant indicators aligned with emerging approaches such as storylines \citep{shepherd2018storylines, sillmann2021event}. These metrics translate abstract resilience concepts into quantifiable features of the possibility space that can guide policy design, infrastructure planning, ecosystem management, and other domains requiring robust decisions under deep uncertainty \citep{walker2013deep, lempert2003shaping}.

The Glass-Plant comparison demonstrates how identical initial conditions and disturbance exposures can produce radically different risk profiles depending on the topology of transformation rules. This finding underscores the limitations of conventional risk assessment approaches that focus on hazard and exposure while treating vulnerability as a static property. Our analysis reveals that system evolution is fundamentally shaped by the balance between degradation and repair processes, and that resilience emerges from the structural properties of the rule set rather than from any single component or parameter.

Future work will develop several extensions of this framework. First, empirical rule calibration from observational data and process models will enable application to real SETS, moving from abstract principles to quantitative prediction. This calibration faces challenges of data sparsity, causal identification, and parameter uncertainty that require integration of machine learning, causal inference, and Bayesian methods \citep{gelman2013bayesian}. Second, coarse-graining techniques for large-scale systems will enable tractable analysis of high-dimensional configuration spaces through dimensional reduction and modular decomposition. Third, exploration of connections to other theoretical frameworks—including thermodynamics \citep{prigogine1972thermodynamics, nicolis1977self}, information geometry \citep{amari2016information}, and category theory—may reveal deeper mathematical structures underlying topological risk. Finally, participatory approaches to rule elicitation and scenario co-production with stakeholders will enhance practical applicability and social legitimacy of the framework.

The multiway causal graph perspective offers not merely a technical tool but a conceptual lens through which to understand how systems navigate uncertain futures. By representing risk as emerging from the structure of possibilities rather than from static properties, this framework aligns with growing recognition that deep uncertainty demands approaches emphasizing robust adaptation over precise prediction \citep{lempert2003shaping, walker2013deep}. As humanity confronts increasingly complex, interconnected challenges from climate change \citep{lenton2008tipping} to technological disruption to social transformation, frameworks that capture the branching, path-dependent, and fundamentally uncertain nature of systemic evolution become essential for navigating toward desirable futures.

\bibliographystyle{plainnat}
\bibliography{references}

\end{document}