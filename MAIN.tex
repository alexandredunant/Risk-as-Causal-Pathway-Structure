\documentclass[12pt]{article}
\usepackage[utf8]{inputenc}
\usepackage[english]{babel}
\usepackage{amsmath}
\usepackage{amssymb}
\usepackage{graphicx}
\usepackage{geometry}
\usepackage{hyperref}
\usepackage{setspace}
\usepackage{float}
\geometry{margin=1in}
\setstretch{1.2}

\title{Risk as Causal Pathway Structure: A Multiway Topological Framework for Systemic Futures}
\author{}
\date{}

\begin{document}
\maketitle

\begin{abstract}
Risk analysis traditionally treats risk as a static quantity derived from hazard, exposure, and vulnerability. This framing fails to capture the deep uncertainty and branching futures of complex Social--Ecological--Technological Systems (SETS). We propose a generative, rule-based framework in which risk emerges not from static probabilities but from the topology of the multiway causal graph---the ensemble of all possible futures generated by local transformation rules. System evolution is described as a causal grammar, with each rule modifying utility, effort, and constraints. We introduce topological metrics such as pathway entropy, Fisher information, and algebraic connectivity to quantify resilience and fragility. Through a synthetic comparison of a degradation-dominant regime (``Glass'') and an adaptive regime (``Plant''), we show how resilience corresponds to structural degeneracy and recombination cycles. This framework provides a new epistemic basis for linking causal structure to decision-relevant scenario diversity, offering principled metrics for robust decision-making under deep uncertainty.
\end{abstract}

\section{Introduction: The Generative Grammar of Risk}

The classical triplet formulation of risk (\emph{hazard} $\times$ \emph{exposure} \(\times\) \emph{vulnerability}) provides useful static intuition but cannot represent the branching, path-dependent futures characteristic of complex SETS. Real systems evolve through local interactions, adaptive decisions, and structural feedbacks. Under deep uncertainty, risk is not a scalar but a geometrical property of the space of possible futures.

We propose a shift from static assessment to a \emph{generative} viewpoint: risk as a property of the \emph{multiway causal graph} generated by the iterative application of local transformation rules. This perspective emphasizes the structure, connectivity, and convergence properties of possible futures, enabling risk evaluation through topological analysis rather than static aggregate measures.

Our approach is inspired by formal rewriting systems, causal graph analysis, network topology, ecological fitness landscapes, and the storyline approach to scenario design. It aims to unify these conceptual threads into a coherent, operationalizable framework for systemic risk.

\subsection*{Epistemic Role of the Framework}

The proposed framework is intentionally multi-layered:

\begin{itemize}
    \item \textbf{Conceptual foundation}: a reformulation of risk as emerging from causal topology.
    \item \textbf{Formal generative model}: a rule-based grammar capable of expressing SETS evolution.
    \item \textbf{Computational instantiation}: when rules are empirically defined, the framework becomes a predictive simulation tool.
\end{itemize}

Our goal is not to impose a single epistemic interpretation but to provide a flexible formalism that supports conceptual reasoning, algorithmic exploration, and empirical grounding.

\section{Formalism: States and Rules}

\subsection{The Configuration Space}

A system state \(\Sigma = (V, E)\) is a graph configuration representing components and their couplings across social, ecological, and technological domains. A state is therefore a high-dimensional configuration rather than a scalar descriptor.

\subsection{Microscopic Transformation Rules}

System evolution is driven by a set of local rules \(R = \{r_1, \ldots, r_n\}\), each of which acts as a rewriting operator:
\[
    r: \Sigma \rightarrow \{\Sigma'_1, \ldots, \Sigma'_k\}.
\]

Each rule is associated with a signature vector:
\[
\delta \vec{r} = (\Delta U, \Delta E, \Delta C),
\]
representing changes in:
\begin{itemize}
    \item \(\Delta U\): utility or performance,
    \item \(\Delta E\): energetic, economic, or operational effort,
    \item \(\Delta C\): constraints, defined as the structural flexibility or restriction of future evolution.
\end{itemize}

\subsubsection*{Interpretation of \(\Delta C\)}
\(\Delta C\) measures how a rule alters the \emph{space of possible futures}. Positive values indicate increased restriction (loss of options), while negative values indicate increased flexibility. This connects to notions of free-energy complexity, ecological fitness landscapes, and path-optional redundancy.

\subsection{Domain-Specific Examples}

\begin{table}[H]
\centering
\begin{tabular}{l l c c c}
\hline
Domain & Rule & $\Delta U$ & $\Delta E$ & $\Delta C$ \\
\hline
Infrastructure & Pipe rupture (edge deletion) & $\downarrow$ & $\downarrow$ & $\uparrow$ \\
Infrastructure & Parallel capacity (edge creation) & $\uparrow$ & $\uparrow$ & $\downarrow$ \\
Ecological & Extinction (node deletion) & $\downarrow$ & $\downarrow$ & $\uparrow$ \\
Ecological & Mutualism (edge creation) & $\uparrow$ & $\uparrow$ & $\downarrow$ \\
Social & Tie dissolution (edge deletion) & $\downarrow$ & $\downarrow$ & $\uparrow$ \\
Social & Bridging (edge creation) & $\uparrow$ & $\uparrow$ & $\downarrow$ \\
\hline
\end{tabular}
\caption{Examples of rule signatures for SETS processes.}
\end{table}

\section{The Action Functional as a Macroscopic Observable}

In contrast to classical variational systems, we do not assume the action determines system trajectories. Instead, we define a simple additive functional that records the cumulative effects of rule applications:

\[
J[\gamma] = \sum_{i=1}^{n} \left( \Delta E(e_i) - \Delta U(e_i) + \Delta C(e_i) \right).
\]

This functional is a \emph{macroscopic summary} of local transformations. It can be interpreted as a discrete analogue of a fitness gradient, but the framework does not rely on any optimization principle. Alternative observables could be used without altering the multiway causal structure.

\section{The Multiway Causal Graph}

Starting from an initial configuration \(\Sigma_0\), recursive application of the rule set generates the multiway causal graph \(G_{\mathrm{MW}}\), where:
\begin{itemize}
    \item nodes are system configurations;
    \item edges represent specific rule applications.
\end{itemize}

This graph encodes the entire accessible future state space.

\subsection{Metrics for Topological Risk}

\subsubsection{Effective Path Diversity (Entropy)}

\[
H = -\sum_{\gamma} P(\gamma) \ln P(\gamma), \quad N_{\mathrm{eff}} = e^H.
\]

High \(N_{\mathrm{eff}}\) indicates a large and diverse set of viable futures.

\subsubsection{Fisher Information (Topological Sensitivity)}

\[
I(\theta) = \sum_{\gamma} P(\gamma) 
\left( \frac{\partial \ln P(\gamma)}{\partial \theta} \right)^2.
\]

Spikes in \(I(\theta)\) indicate topological phase transitions.

\subsubsection{Spectral Gap (Structural Connectivity)}

\[
\lambda_2 \text{ of the Laplacian } L
\]

measures algebraic connectivity. Collapse corresponds to $\lambda_2 \to 0$.

\subsection{Decision Relevance: Linking to Storylines}

Topological metrics map naturally to the Storyline approach to scenario design:
\begin{itemize}
    \item \(N_{\mathrm{eff}}\): number of structurally distinct storylines,
    \item \(\lambda_2\): feasibility of transitions between storylines,
    \item \(I(\theta)\): sensitivity of storyline structure to forcing changes.
\end{itemize}

This provides actionable, decision-relevant indicators of systemic flexibility.

\section{Case Study: Glass and Plant Regimes}

We compare two archetypal rule regimes:

\begin{itemize}
    \item \textbf{Glass}: dominated by deletion rules; degradation without repair.
    \item \textbf{Plant}: balanced deletion and creation; local repair via transitivity.
\end{itemize}

\subsection{Topological Divergence}

\emph{Glass}: partial mesh topology collapsing to a unique sink.  
\emph{Plant}: rich recombination cycles producing structural degeneracy.

\begin{figure}[H]
    \centering
    \includegraphics[width=\textwidth]{multiway_futures_comparison.png}
    \caption{Comparative evolution of the Glass (left) and Plant (right) multiway causal graphs. The Glass system exhibits a collapsing topology where all paths converge to a single failure state. The Plant system exhibits a rich, recombining topology with high degeneracy, maintaining multiple viable configurations despite identical initial hazard and exposure.}
    \label{fig:comparison}
\end{figure}

\begin{figure}[H]
    \centering
    \includegraphics[width=0.8\textwidth]{topological_risk_table.png}
    \caption{Quantification of the topological analysis of the two systems "Glass" and "Plant".}
    \label{fig:metrics}
\end{figure}

\subsection{Causal Invariance and Closure}

\textbf{Glass}: trivial closure; all paths converge to collapse.  
\textbf{Plant}: nontrivial closure; multiple trajectories reunite into functional states.

\subsection{Non-Closure and Lock-In}

Systems lacking repair rules form purely divergent trees:
\[
R_a(R_b(\Sigma)) \neq R_b(R_a(\Sigma)).
\]
This induces hysteresis and irreversible loss of future options.

\section{Conclusion}

Risk should be understood as a property of the topology of possible futures, not as a static scalar. The multiway causal graph provides a generative space in which resilience corresponds to structural degeneracy and adaptive reconvergence, while fragility corresponds to narrowing connectivity and lock-in. The proposed framework integrates conceptual, formal, and computational roles and provides decision-relevant metrics aligned with emerging approaches such as storylines. Future work will develop empirical rule calibration and coarse-graining techniques for large-scale SETS.

\end{document}
